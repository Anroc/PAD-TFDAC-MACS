%%%%%%%%%%%%%%%%%%%%%%%%%%%%%%%%%%%%%%%%%
% Masters/Doctoral Thesis 
% LaTeX Template
% Version 2.5 (27/8/17)
%
% This template was downloaded from:
% http://www.LaTeXTemplates.com
%
% Version 2.x major modifications by:
% Vel (vel@latextemplates.com)
%
% This template is based on a template by:
% Steve Gunn (http://users.ecs.soton.ac.uk/srg/softwaretools/document/templates/)
% Sunil Patel (http://www.sunilpatel.co.uk/thesis-template/)
%
% Template license:
% CC BY-NC-SA 3.0 (http://creativecommons.org/licenses/by-nc-sa/3.0/)
%
%%%%%%%%%%%%%%%%%%%%%%%%%%%%%%%%%%%%%%%%%

%----------------------------------------------------------------------------------------
%	PACKAGES AND OTHER DOCUMENT CONFIGURATIONS
%----------------------------------------------------------------------------------------

\documentclass[
11pt, % The default document font size, options: 10pt, 11pt, 12pt
%oneside, % Two side (alternating margins) for binding by default, uncomment to switch to one side
english, % ngerman for German
singlespacing, % Single line spacing, alternatives: onehalfspacing or doublespacing
%draft, % Uncomment to enable draft mode (no pictures, no links, overfull hboxes indicated)
%nolistspacing, % If the document is onehalfspacing or doublespacing, uncomment this to set spacing in lists to single
%liststotoc, % Uncomment to add the list of figures/tables/etc to the table of contents
%toctotoc, % Uncomment to add the main table of contents to the table of contents
%parskip, % Uncomment to add space between paragraphs
%nohyperref, % Uncomment to not load the hyperref package
headsepline, % Uncomment to get a line under the header
%chapterinoneline, % Uncomment to place the chapter title next to the number on one line
%consistentlayout, % Uncomment to change the layout of the declaration, abstract and acknowledgements pages to match the default layout
]{MastersDoctoralThesis} % The class file specifying the document structure

\usepackage[utf8]{inputenc} % Required for inputting international characters
\usepackage[T1]{fontenc} % Output font encoding for international characters

\usepackage{mathpazo} % Use the Palatino font by default

\usepackage[backend=bibtex,style=numeric,natbib=true]{biblatex} % Use the bibtex backend with the authoryear citation style (which resembles APA)

\addbibresource{multi-authority-abe.bib} % The filename of the bibliography

\usepackage[autostyle=true]{csquotes} % Required to generate language-dependent quotes in the bibliography
\usepackage{makecell}
\renewcommand\theadfont{\bfseries}
\renewcommand\theadalign{bl}
\renewcommand\cellalign{cl}

\usepackage{acro}
\acsetup{first-style=short}
\usepackage{amsmath}
\usepackage[titletoc,toc,title]{appendix}
\usepackage{listings}
\lstset{basicstyle=\ttfamily\scriptsize}


\usepackage[textsize=footnotesize]{todonotes} % for todo notes
% define new command for \req (bold and italic)
\newcommand{\req}[1]{\textbf{\textit{#1}}}

% use \todo[inline] per default
\let\originaltodo\todo
\renewcommand{\todo}{\originaltodo[inline]}

\newcommand{\question}[1]{\originaltodo[color=green, inline]{#1}}

\newcommand{\name}{\ac{PAD-TF-DAC-MACS} }
\newcommand{\fullname}{Practically Applied Distributed Two-Factor authentication Data Access Controll for Multi-Authority Cloud Storage systems}


%----------------------------------------------------------------------------------------
%	MARGIN SETTINGS
%----------------------------------------------------------------------------------------

\geometry{
	paper=a4paper, % Change to letterpaper for US letter
	inner=2.5cm, % Inner margin
	outer=3.8cm, % Outer margin
	bindingoffset=.5cm, % Binding offset
	top=1.5cm, % Top margin
	bottom=1.5cm, % Bottom margin
	%showframe, % Uncomment to show how the type block is set on the page
}

%----------------------------------------------------------------------------------------
%	THESIS INFORMATION
%----------------------------------------------------------------------------------------

\thesistitle{Practical Multi-Authority Attribute-Based Encryption for Secure Cloud Storage Systems} % Your thesis title, this is used in the title and abstract, print it elsewhere with \ttitle
\supervisor{Prof. Dr. Florian \textsc{Tschorsch}} % Your supervisor's name, this is used in the title page, print it elsewhere with \supname
\examiner{} % Your examiner's name, this is not currently used anywhere in the template, print it elsewhere with \examname
\degree{Master of Science} % Your degree name, this is used in the title page and abstract, print it elsewhere with \degreename
\author{Marvin \textsc{Petzolt}} % Your name, this is used in the title page and abstract, print it elsewhere with \authorname
\addresses{} % Your address, this is not currently used anywhere in the template, print it elsewhere with \addressname

\subject{Computer Sciences} % Your subject area, this is not currently used anywhere in the template, print it elsewhere with \subjectname
\keywords{} % Keywords for your thesis, this is not currently used anywhere in the template, print it elsewhere with \keywordnames
\university{\href{https://tu-berlin.de}{Technical Unisersity Berlin}} % Your university's name and URL, this is used in the title page and abstract, print it elsewhere with \univname
\department{\href{https://dsi.tu-berlin.de}{Distributed Systems Infrastructure}} % Your department's name and URL, this is used in the title page and abstract, print it elsewhere with \deptname
\group{\href{https://www.eecs.tu-berlin.de/menue/fakultaet_iv/?no_cache=1}{\textsc{Faculty} IV,}} % Your research group's name and URL, this is used in the title page, print it elsewhere with \groupname
\faculty{\href{https://www.eecs.tu-berlin.de/menue/fakultaet_iv/?no_cache=1}{\textsc{Faculty} IV}} % Your faculty's name and URL, this is used in the title page and abstract, print it elsewhere with \facname

\AtBeginDocument{
\hypersetup{pdftitle=\ttitle} % Set the PDF's title to your title
\hypersetup{pdfauthor=\authorname} % Set the PDF's author to your name
\hypersetup{pdfkeywords=\keywordnames} % Set the PDF's keywords to your keywords
}

\DeclareAcronym{AA}{
  short = AA ,
  long  = Attribute Authority ,
  class = abbrev
}

\DeclareAcronym{CA}{
  short = CA ,
  long  = Central Authority ,
  class = abbrev
}

\DeclareAcronym{CSP}{
  short = CSP ,
  long  = Cloud Storage Provider ,
  class = abbrev
}

\DeclareAcronym{ABE}{
  short = ABE ,
  long  = Attribute-Based Encryption ,
  class = abbrev
}

\DeclareAcronym{CP-ABE}{
  short = CP-ABE ,
  long  = Ciphertext policy Attribute-Based Encryption ,
  class = abbrev
}

\DeclareAcronym{KP-ABE}{
  short = CP-ABE ,
  long  = Key policy Attribute-Based Encryption ,
  class = abbrev
}

\DeclareAcronym{HABE}{
  short = HABE ,
  long  = Hirachical Attribute-Based Encryption ,
  class = abbrev
}

\DeclareAcronym{DABE}{
  short = CP-ABE ,
  long  = Decentralized Attribute-Based Encryption ,
  class = abbrev
}

\DeclareAcronym{MA-ABE}{
  short = MA-ABE ,
  long  = Multi-Authority Attribute-Based Encryption ,
  class = abbrev
}

\DeclareAcronym{DAC-MACS}{
  short = DAC-MACS ,
  long  = Effective Data Access Control for Multi-Authority Cloud Storage Systems ,
  class = abbrev
}

\DeclareAcronym{TF-DAC-MACS}{
  short = TF-DAC-MACS ,
  long  = Two-Factor DAC-MACS ,
  class = abbrev
}


\begin{document}

\frontmatter % Use roman page numbering style (i, ii, iii, iv...) for the pre-content pages

\pagestyle{plain} % Default to the plain heading style until the thesis style is called for the body content

%----------------------------------------------------------------------------------------
%	TITLE PAGE
%----------------------------------------------------------------------------------------

\begin{titlepage}
\begin{center}

\vspace*{.06\textheight}
{\scshape\LARGE \univname\par}\vspace{1.5cm} % University name
\textsc{\Large Master Thesis}\\[0.5cm] % Thesis type

\HRule \\[0.4cm] % Horizontal line
{\huge \bfseries \ttitle\par}\vspace{0.4cm} % Thesis title
\HRule \\[1.5cm] % Horizontal line
 
\begin{minipage}[t]{0.4\textwidth}
\begin{flushleft} \large
\emph{Author:}\\
\authorname % Author name - remove the \href bracket to remove the link
\end{flushleft}
\end{minipage}
\begin{minipage}[t]{0.4\textwidth}
\begin{flushright} \large
\emph{Supervisor:} \\
\supname % Supervisor name - remove the \href bracket to remove the link  
\end{flushright}
\end{minipage}\\[3cm]
 
\vfill

\large \textit{A thesis submitted in fulfillment of the requirements\\ for the degree of \degreename}\\[0.3cm] % University requirement text
\textit{in the}\\[0.4cm]
\groupname\\\deptname\\[2cm] % Research group name and department name
 
\vfill

{\large \today}\\[4cm] % Date
%\includegraphics{Logo} % University/department logo - uncomment to place it
 
\vfill
\end{center}
\end{titlepage}

%----------------------------------------------------------------------------------------
%	DECLARATION PAGE
%----------------------------------------------------------------------------------------

\begin{declaration}
\addchaptertocentry{\authorshipname} % Add the declaration to the table of contents
\noindent I, \authorname, declare that this thesis titled, \enquote{\ttitle} and the work presented in it are my own. I confirm that:

\begin{itemize} 
\item This work was done wholly or mainly while in candidature for a research degree at this University.
\item Where any part of this thesis has previously been submitted for a degree or any other qualification at this University or any other institution, this has been clearly stated.
\item Where I have consulted the published work of others, this is always clearly attributed.
\item Where I have quoted from the work of others, the source is always given. With the exception of such quotations, this thesis is entirely my own work.
\item I have acknowledged all main sources of help.
\item Where the thesis is based on work done by myself jointly with others, I have made clear exactly what was done by others and what I have contributed myself.\\
\end{itemize}
 
\noindent Signed:\\
\rule[0.5em]{25em}{0.5pt} % This prints a line for the signature
 
\noindent Date:\\
\rule[0.5em]{25em}{0.5pt} % This prints a line to write the date
\end{declaration}

\cleardoublepage

%----------------------------------------------------------------------------------------
%	QUOTATION PAGE
%----------------------------------------------------------------------------------------

%\vspace*{0.2\textheight}

%\noindent\enquote{\itshape Thanks to my solid academic training, today I can write hundreds of words on virtually any topic without possessing a shred of information, which is how I got a good job in journalism.}\bigbreak

%\hfill Dave Barry

%----------------------------------------------------------------------------------------
%	ABSTRACT PAGE
%----------------------------------------------------------------------------------------

\begin{abstract}
\addchaptertocentry{\abstractname} % Add the abstract to the table of contents
Commonly a group of users want to share a ciphertext, they need to agree on a shared key. In order to ensure forward and backward secrecy, this shared key needs to be updated and distributed to every member everytime a user leaves or joins the group. Classical encryption methods encounter their limits, in which this approach is no longer feasible for a large number of users. In this work the field of attribute based encryption (ABE) have been analyzed to solve this issue for a secure cloud storage environment. Different ABE schemes haven been compared for their applicability in Bdrive. Finally, the implementation of the adapted scheme was evaluated for performance, scalability and security, showing that secure group communication using ABE can achieve sub-linear rekeying overhead. 
\end{abstract}

%----------------------------------------------------------------------------------------
%	ACKNOWLEDGEMENTS
%----------------------------------------------------------------------------------------

% \begin{acknowledgements}
% \addchaptertocentry{\acknowledgementname} % Add the acknowledgements to the table of contents
%The acknowledgments and the people to thank go here, don't forget to include your project advisor\ldots
%\end{acknowledgements}

%----------------------------------------------------------------------------------------
%	LIST OF CONTENTS/FIGURES/TABLES PAGES
%----------------------------------------------------------------------------------------

\tableofcontents % Prints the main table of contents

% \listoffigures % Prints the list of figures

% \listoftables % Prints the list of tables

%----------------------------------------------------------------------------------------
%	ABBREVIATIONS
%----------------------------------------------------------------------------------------

%\begin{abbreviations} % Include a list of abbreviations (a table of two columns)

%\printacronyms

%\end{abbreviations}

%----------------------------------------------------------------------------------------
%	THESIS CONTENT - CHAPTERS
%----------------------------------------------------------------------------------------

\mainmatter % Begin numeric (1,2,3...) page numbering

\pagestyle{thesis} % Return the page headers back to the "thesis" style

% Introduction / Background  / Problem description
\section{Introduction}
\label{sec:introduction}
\todo{
    * **Introduction**\\
    * Ruff problem describtion\\
    * How ABE can solve this\\
    * How ABE works\\
    * Short history summary of ABE\\
    * What is expected of this work\\
    * What are the targets
}
\lettrine[nindent=0em,lines=3]{I}n public-key cryptography each user is identified by his unique public and private key pair. Peer-to-peer communication works well with this scheme, but as soon as an encrypted content needs to be accessed by multiple participants, the data owner has to encrypt the content for each user explicitly. This results in many encrypted versions of the same  file, each secured by a different public key. Such scheme does not scale in the secure cloud storage domain. Here often data holders want to share the same content with many other users at the same time.

We reach the point where the classical public-key end-to-end encryption scheme does not scale anymore. We would like to employ an encryption scheme which has a constant number of access keys regardless of the number of participants.
\todo{extend}

\subsection{Secure cloud storage system}
%Objective
\todo{	
	* How does rekeying in bdrive work currently?\\
	* What security targets are covered\\
	* What is adventagous \\
	* Waht is disadventagous
	}
In the following we will describe the encryption scheme currently enrolled in Bdrive and give a rough overview of the existing solutions in the attribute-based encryption domain.

\subsection{Background}
\begin{figure*}[!ht]
\centering
    \includegraphics[width=0.8\linewidth]{img/bdrive1.png}\par 
    \caption{Client uploads an encrypted file to the CSPs and the file key and public key to Bdrive.}
    \label{fig:filekey}
\end{figure*}
\begin{figure*}[!ht]
\centering
    \includegraphics[width=0.8\linewidth]{img/bdrive2.png}\par
    \caption{Client A grants Client B access to the uploaded file by re-keying the file key}
    \label{fig:rekey}
\end{figure*}

Bdrive is a secure cloud storage which splits up files in smaller chunks that are saved separately on different cloud storage provider (CSP). To ensure end-to-end encryption a Bdrive client encrypts locally each of its chunks with a one-time symmetric key that is then encrypted under its own public key. This encrypted key is called a file key and it is uploaded to the Bdrive server where it is stored securely (see figure \ref{fig:filekey}).

Since each device of the same user has a different private-public key pair, the device is in charge of making the file keys accessible for a new device. This will be done by downloading each file key for the respective file, receiving the public key of the new device, decrypting the file key with its own private key, encrypting it again with the public key of the new device and finally, uploading the new file key to the Bdrive server. This process will be called re-keying (see figure \ref{fig:rekey}).

%The formula \ref{eq:rekey} describes the number of file keys Bdrive that need to be stored for each shared files between $U$ users, where each user $u_{i \in U}$ has $u_d$ devices.

%\begin{equation}
%n = \sum_{i}{d_i}
%\label{eq:rekey} 
%\end{equation}

%Lets construct an example where the manager of a company wants to create a shared folder with all company employees. It is a medium sized company with $50$ employees. Lets assume that at least haft of them have two Bdrive clients running. The manager wants to upload the $250$ photos of the last company trip.
%We end up by computing $3/2 * 50 * 250 = 18750$ file keys and for every new file uploaded $75$ new file keys need to be uploaded. 

\subsection{Problem describtion}
% Bdrive rekeying
% Motivation and Problem description
\todo{find better title}

% File upload and file key creation
In Bdrive each device of a user generates a new RSA key pair on registaion. The fingerprint (hash) of the public key identifies the device uniquly. To save a file in the cloud the device fist encrypes the file symmetrically with the so called "filekey". The filekey equals the hash of the plain file content and so ensures tamperproofness and integirty on decryption. Since Bdrives supports end-to-end encryption the server should never be able to decrypt the file by itself. The device encrypts the filekey asymmetrically with its own public key and uploads the filekey to the Bdrive server were it is stored securly. The encrypted file is uploaded to the different cloud storage provider. 

% Access file
If the user wants to access a file locally, the devices requests the encrypted filekey from the server, downloads the file chunks from the cloud storage providers, decrypts the file key with his privat key and finally decrypts the file with the filekey. 

% Process for multibe devices
So far we saw how the encryption process is handled for a single device. However, this process turns out to be much more complex in computation in a multible devices setting.  If a user registers more then one device the exisiting data needs to be syncronized to the new device. The server nodifies the existing device for the new registered device and the public key of the new device is downloaded. Now, each filekey for each file of the user needs to be downloaded from the server and decrypted. To make them accessable for the new devices, they are encrypted with the new devices public key and uploaded again to the server. The new device can now start to download and decrypt the filekeys as decibed previously. 

% Adventages
If a new devices joins the share we need to make the existing file keys avaiable to the new devices which results in $O(f)$ additional encryptions, messages and keys. However, a big adventage of this scheme is that we don't have to anything if a member leaves the group. We simply remove all the filekeys belonging to the left device and do not further encrypt new uploaded files for this device. However, this scheme benefits from implicit forward secrecy and backward secrecy. This is due to each filekey is encrypted individually for each entity so for each new file a new secret is generated. Which means that new members do not have any knowledge about the previously exchanged messages until they joind the group, formally known as backward secrecy. Forward secrecy describes the process is ensured. Backwards secrecy, which desribes the process that after a member is removed or vollenturally left the group he got no further access to the group communication anymore.

The backward secrecy contstain, while being an imported security feature, will explicitly be broken by clients. If a data owner invites a new member into a group he wishes that the new member receives all previous uploaded information. Bdrive is ensuring this by explicitly reencyrpting all file keys for the new device. 
%But there are better ways to share content more efficiently while disrepecting the constrains of backward secrey but still mainting forward secrecy. 

% Disadventages in shares
Notice, that this approach does not scalable for a large number of devices. In addition, a cloud storage system usally provides the possibility to share the files with other users. Which means that each device of the user invited to a share needs to have an own encrypted file key issued. This process scales with $O(n * f)$ keys. Were $n$ donates the number of devices onvolved in the sharing and $f$ the number of file versions in the share.\footnote{Each file consit of many file versions. Each file version needs an own file key since the content changes each time the file is updated.} Futher, we need to send $O(n * f)$ messages to send each device its personal filekey and we need to encrypt each filekey $n$ times which also results in $O(n * f)$ encryptions in total. 

% Problem
The number of file keys, that need to be maintained in Bdrive, grows linearly with the number of devices. In addition Bdrive allows to share files between different users. For each device of each user involved in a share a new file key has to be created and maintained. 

\subsection{Contribution}
\todo{
    * What are the targets\\
    * What makes this work stand out\\
    * What we can gain from this work
}
The target of this work is to find a more scalable solution to the currently enrolled re-keying scheme in Bdrive. The new scheme needs to satisfy the requirements of section \ref{sec:requirements} with respect to multi-company management (each company administrates only its own domain) and inter-company file sharing (it is possible to share files across companies), as well as it should be practically applicable in the real world.
\todo{extend}

% Related overview schemes and cloud storage evaluation
\section{Related Work}
\todo{
	* **Related work**\\
	* Of paper did something similar to this\\
	* Comparison paper\\
	* Not \ac{ABE} paper
}

\todo{
	Cite \cite{lee2013survey} and similar overview paper
}

% \subsection{Attribute-Based Encryption}
% Attribute Based Encryption (\ac{ABE}), first introduced by Sahai and Waters \cite{sahai2005fuzzy}, is a cryptographic encryption scheme which uses attributes of a user as keys for encryption. This enables the data owner to craft a ciphertext over chosen attributes that can only be decrypted by any entity that holds a super set of matching attributes. Further, it is possible to embed a complex access structure (e.g. access trees) inside the cipher text, where each node contains \textit{\ac{AND}} or \textit{\ac{OR}} threshold gates. This approach is known as Ciphertext-Policy Attribute-Based Encryption (\ac{CP-ABE}) \cite{bethencourt2007ciphertext}. 
% It is also possible to do it the other way around: Associate the user's key with an access policy, formally known as Key-Policy Attribute Based Encryption (\ac{KP-ABE}).\cite{goyal2006attribute}. 
% %Now the encryptor needs only to encrypt a given plain text with the public key of specific attributes so that only users who hold the right keys are able to decrypt the cipher text.

% Both approaches are limited to one attribute authority (\ac{AA}). So only one trusted entity is in charge of issuing attributes and their matching key pairs. However, in the real world we would like to distribute the issuing of attributes over different authorities. A new scheme is needed where different attribute authorities cooperate and communicate across different domains.  

% % The basic idea is, as proposed in \cite{chase2007multi}, to construct for each users a polynomial of degree $d-1$, where $d$ donates the number of attributes in our system. 

% Multi-Authority Attribute Based Encryption (\ac{MA-ABE}), first proposed by Chase 2007 \cite{chase2007multi}, allows multiple attribute authorities to maintain different attribute domains. To ensure collusion resistance between users, a trusted  Central Authority (\ac{CA}) was introduced to assign each user an unique identifier (\ac{UID}) and making the decryption process depending on this \ac{UID}. Its disadvantage is the \ac{CA}'s global decryption power and therefore the necessity to remain trusted. 
% %To prevent collusion between user in a multi authority setting, the challenge for each user needs to be individual. But it still needs to be ensured that the encryption of a message is independent of any user specific identifier, since the encryption progress should sourly depend on the attribute set known to the system.
% %To mitigate this problem a global identifier (\ac{GID} or \ac{UID}) per users was introduced that is shown to each attribute authority (\ac{AA}) to receive the corresponding private key for the users attributes. 
% %The central authority (\ac{CA}) now has to make sure that the user dependent challenge results in a global decryption key to decrypt the message.
% %In fact the \ac{CA} has to be trusted since it computes the users private keys in such a way that on decryption it reveals the global decryption key. 
% Chase \textit{et al.} addressed this issue in \cite{chase2009improving} by distributing the global secret  master key generation over the \ac{AA}s. However, since the global master secret is computed on system initialization, no further \ac{AA}s can be added afterwards without rebooting the system. Also the scheme did not support attribute revocation which rendered it practical not applicable.
% % Each authority uses this seed in combination with a pseudo random function to deterministically create the users private key. Since the \ac{CA} possesses the same seed, same pseudo random function and issues the users \ac{GID}, it can also compute the same private key as issued by the corresponding \ac{AA}. So it can ensure that on decryption the keys add up to reveal the global decryption key. This scheme has a major issue: The \ac{CA} has global decryption power. 

% % Chase notices this problem as well and tried to mitigate it by decentrializing the seed generation \cite{chase2009improving}.  First user attributes could not easly be revoked\footnote{Chase proposed to assign each user a range of \ac{GID}s, so that a user can migrate to the next \ac{GID} on revocation. Since the \ac{GID} range is finite, this solution does not scale.}. Second \ac{AA}'s cant be added on runtime without reissuing each user his keys. And last, the attribute universe is finite (no large-universe). Further we noticed that if a failure of one \ac{AA} renders the system unresponsible, since a cyphertext includes attributes from each authority. 

% Data Access Control for Multi-Authority Cloud Storage Systems (\ac{DAC-MACS}) \cite{yang2013dac} is a scheme that tackles both of this issues. Each authority receives its own ciphertext that depends solely on the attributes issued in this domain. While Chase managed to maintain the "one-plaintext-one-ciphertext" ratio, \ac{DAC-MACS} needs to create $k$ ciphertexts: One per \ac{AA}.\footnote{If the ciphertext does not require any attributes of an specific authority it does not have to create a ciphertext for this domain.} It does not require any coordination between authorities which enables to add new \ac{AA}s at runtime without recreating the user keys. This scheme also includes features for efficient revocation while it claims to maintain forward and backward secrecy. 

% \ac{DAC-MACS} is not collusion resistance on attribute revocation under the active attack model. The scheme \ac{NEDAC-MACS} (New Extended \ac{DAC-MACS}) shows and solves this vulnerability \cite{wu2017security}. Recent studies introduce a more efficient, scalable and secure approaches such as \ac{MAACS} \cite{li2016secure} and \ac{TF-DAC-MACS} (Two-Factor \ac{DAC-MACS})\cite{li2017two}. The \ac{DAC-MACS} family is further known for the proxy decryption technique, first introduced by \cite{li2013matrix}, where a honest-but-curious server helps the user on decryption.

% Priscilla and Nagarajan 2018 \cite{nagarajan2018overview} conducted an analysis of the \ac{DAC-MACS} family. In addition to that work, this thesis will focus on alternative approaches to improve scalability of secure cloud storage systems and analyze the field of \ac{ABE} and \ac{MA-ABE} more deeply.

% Requirements
\chapter{Background}
\todo{write introdction}
\todo{check whether following sections make sense }

\section{Secure Cloud Storage System}
In the past 20 years we witnessed a shift of resource consumption on the user side to outsourcing more and more resources into the cloud. Naturally, users moved their files into cloud storage systems. This helped companies to maintain an ecosystem around the user helping him to synchronize his files onto different devices.  

With increasing transparency raised also the concern that it was not clear anymore who could access the private files especially when they are transmitted and stored on oversee servers. Bdrive, a secure cloud storage system, committed to not export files and data to other countries. It splits up files in smaller chunks that are saved separately on different cloud storage provider (\ac{CSP}). To ensure end-to-end encryption a Bdrive client encrypts locally each of its chunks with a one-time symmetric key that is then encrypted under its own public key. This encrypted key is called a file key and it is uploaded to the Bdrive server where it is stored securely, as shown in Figure \ref{fig:filekey}.

\section{Background}
\label{sec:background}
\begin{figure*}[!ht]
\centering
    \includegraphics[width=1\linewidth]{img/bdrive1.png}\par 
    \caption{Device uploads an encrypted file to the \ac{CSP}s and the file key and public key to Bdrive.}
    \label{fig:filekey}
\end{figure*}
\begin{figure*}[!ht]
\centering
    \includegraphics[width=1\linewidth]{img/bdrive2.png}\par
    \caption{Device A grants Device B access to the uploaded file by re-keying the file key}
    \label{fig:rekey}
\end{figure*}
Since each device of the same user has its own private-public key pair, an existing device is in charge of making its accessible file keys available for a new device. This will be done by downloading each file key for the respective file, receiving the public key of the new device, decrypting the file key with its own private key, encrypting it again with the public key of the new device and finally, uploading the new file key to the Bdrive server. This process will be called re-keying, as shown in Figure \ref{fig:rekey}. 

% File upload and file key creation
In Bdrive each device of a user generates a new \ac{RSA} key pair on registration. The fingerprint (SHA-1 or MD5 hash) of the public key identifies the device uniquely. To save a file in the cloud the device first encrypts the file symmetrically with the so called "filekey". The filekey equals the hash of the plain file content and so ensures tamperproofness and integrity on decryption. End-to-end encryption implies that the server should never be able to decrypt the file by itself. To enforce this the device encrypts the filekey asymmetrically with its own public key before uploading the filekey to the Bdrive server where it is stored securely. In Bdrive, the encrypted file chunks are uploaded to the different cloud storage provider (\ac{CSP}). 

% Access file
If the user wants to access a file locally, the devices requests the encrypted filekey from the server and downloads the file chunks from the \ac{CSP}s. Locally, it decrypts the filekey with his private key and finally deciphers the assembled file with the decrypted filekey. 

% Process for multibe devices
So far the encryption process for a single device has been outlined. However, this process turns out to be much more computationally complex in a multiple device setting.  If a user registers more then one device the existing data needs to be synchronized to the new device. The server notifies the existing device for the newly registered one and the public key of the new device is downloaded. Now the existing device needs to download each filekey for each file of the user and decrypts it. The synchronization is finished by encrypting the filekeys with the new devices public key and uploaded again to the server. The new device can now start to download and decrypt the file chunks as described previously.

Usually in cloud storage systems we also have the concept of groups. They describe a collection of clients which share data between them. For that they form a so called \textit{share}. To express the overhead of joining or leaving a share the following notation will be used: $f$ and $n$ donate the number of filekeys and the number of devices in a share respectively.

% Adventages
If a new devices joins a share in Bdrive a device needs to make the existing file keys available to the new devices which results in $f$ additional encryptions, messages and keys. However, a big advantage of this scheme is that forward secrecy\footnote{Forward Secrecy: The left entity will have no knowledge about future shared content.} will not produce additional overhead. Here, only all the filekeys belonging to the left device are removed and the other devices do not further encrypt new uploaded files for this device anymore. The backward secrecy constrain, while being an imported security feature, will explicitly be broken by clients. This is due to the feature, that data owner can invite a new member into a group and the new member accesses all previous uploaded content. Bdrive is ensuring this by explicitly reencrypting all file keys for the new device. 

\section{Problem Description}
% Bdrive rekeying
% Motivation and Problem description

% Disadventages in shares
Notice, that this approach is not scalable for a large number of devices. To showcase this, the use case of creating a new share is analyzed. Each device invited to a share needs to have an own version of the file key, encrypted with the devices public key, issued. This process scales with $O(n * f)$ keys. 
%Were $n$ donates the number of devices involved in the sharing and $f$ the number of file versions in the share.\footnote{Each file consist of many file versions. Each file version needs an own file key since for each content change a new file is updated.}
Further, the same number of messages containing the encrypted file key need to be send and each filekey needs to be encrypted $n$ times which also results in $O(n * f)$ encryptions in total. 

To make the overhead more clear, lets assume a manager of a company with 50 employees who wants to create a company wide share. Each of the employees has at least two devices (say one web view client and a desktop client). To upload the latest presentation the device of the manager know has to create $50 * 2 = 100$ filekeys to upload, $100-1$ messages, containing encrypted file key for each device, to distribute and $100$ encryptions to make. Even worst for each new presentation upload another $100$ filekeys need to be maintained. In a large scale company this overhead becomes unmaintainable when the number of $10.000$ filekeys are exceeded. \footnote{Running 'openssl speed rsa1024' on an Intel(R) Core(TM) i7-6500U CPU @ 2.50GHz takes on average 0.000113s for encryption. Multiplied by 10.000 the second mark is reached.} With increasing computing power this problem can be compensated but not prevented.

\section{Attribute-Based encryption}
\label{sec:attribute-based-encryption}
\textit{Attribute-Based Encryption} (\ac{ABE}) defines users over attributes and attribute keys rather then its individual public key. Since users are not unique among their attribute set it is possible to reduce the number of needed keys of a share to the number of attributes necessary to describe the group completely. 

In the previous example the manager would only need to encrypted the presentation one time: With the public key of the attribute “working in company”. In this participial use case only $1$ encryption is done, $1$ message is distributed using multi-cast to transmit same encrypted file to all clients, and $1$ file key is created. 

Of course this advantage does come at some cost. While classical encryption schemes provide complete End-to-End encryption, ABE needs a \textit{Attribute Authority} (\ac{AA}) to issue attribute and attribute keys in the first place. This authority has global decryption power in the administered domain. For that reason the attributes need to be split up into different domains, each managed by an own AA. In the use case of cloud storage systems for business it makes sense to setup an AA for each company. 

With ABE comes the advantage of defining access policies for cipher texts. This policies consist of boolean formulas with attribute values as inputs. Only if a user satisfies the given policy he is able to decipher the cipher text. The use of this access formulas helps to eliminate authorization checks, since clients can either decrypt the file and are therefor allowed to access the plaintext or they are not able to decrypt and so do not satisfy the given policy. 

% In our use case inter company sharing becomes tricky. Now different AAs need to cooperate to create cipher text policies containing attributes of the whole ecosystem.  

ABE scales with the number of attributes contained in a cipher text. As long as a group can be described with less attributes then there are members in this group, a better performance of ABE compared to the current scheme in Bdrive can be achieved. 

ABE comes in many different favors. The most important for this work are Single-Authority Attribute-Based Encryption (normal \ac{ABE}) and Multi-Authority Attribute-Based Encryption (\ac{MA-ABE}). 
For the single authority use case three entities are defined:

\begin{enumerate}
	\item \textbf{Attribute Authority (\ac{AA});} An attribute authority issues user their unique identifier (\ac{UID}), attributes and their respective attribute secret keys. 
	On revocation the AA will need to update the users secret keys as well as the ciphertext encrypted with the revoked attribute key. 
	Since an AA always generates the private attribute key to derivate a new user specific secret key it has global decryption power. That’s why the single-authority use case the AA is assumed to be an trusted entity. 
	\item \textbf{Cloud Storage Provider (\ac{CSP});} The cloud storage provider are assumed to be untrusted but they still follow the protocol. That’s why they only receive encrypted data. They only purpose is to store the ciphertext and make them permanently available. Since the authorization to each ciphertext is directly embedded into the cipher-text access policy, the download of the encrypted file does not need any authentication checks.
	\item \textbf{Users/Data Owners;} Users exist in two groups: Revoked and non-revoked. Non-revoked users try to collude with each other to get a higher level of decryption power. They download the files of the \ac{CSP} and try to decrypt them. Only if they attribute set matches the policy of the ciphertext they will be able to decrypt the file. 
	Revoked users, on the other hand, try to still decipher ciphertext. In some cases they try to collude with non-revoked user to intercept the key update key to restore their decryption rights. 
	User are in general untrusted.

	Users (some schemes differentiate between decrypting users and encrypting data owners) want to encrypt content with a specific access policy. To do so they use the public available public attribute keys pinned on the bulletin board of the \ac{AA}. They do not have to know anything about the receiving user or user groups in the system. After encryption they update the encrypted content to the \ac{CSP}.
\end{enumerate} 

MA-ABE states the following adaptation and additions:

\begin{enumerate}
	\item \textbf{Attribute Authority (\ac{AA});} In the system of MA-ABE there are multiple AAs. Each administrating their own domain. AAs are usually assumed to be honest-but curios. They distrust each other and try to decrypt files of other domains. A big goal for MA-ABE is to achieve inter-authority communication so that cipher-texts can ce encrypted using different attributes of different domains. An AA still bootstraps its users, but can also issue external users attributes.
	\item \textbf{Certificate/Central Authority (\ac{CA});} The purpose of the \ac{CA} is to issue user their global identifier (\ac{GID}). This GID is unique among the whole user universe and identify a user uniquely against each AA. Further, it bootstraps the different \ac{AA}s. The \ac{CA} usually remains trusted but. In many MA-ABE schemes the CA has global decryption power. The CA can also act as an \textit{Certificate Authority} to make users as valid participants in the system. If central authority and certificate authority have different trust levels, they are modeled as different entities.
	\item \textbf{Server;} Some schemes additional introduce a server. Its task is it to help user with proxy re- and decryption. If an \ac{AA} broadcasts a revocation of an attribute, the server downloads all related ciphertexts from the \ac{CSP} to update them with the new attribute. 
	The thread model for the server is honest-but-curious or even untrusted. Please note, that the \ac{CA} and the server are two separated entities that do not cooperate.
\end{enumerate} 

Users and CSP remain unchanged in the MA-ABE case.

\section{Thread Model}
\label{sec:thread-model}
MA-ABE describes six entities which will have different trust levels in the design. In the following the certificate authority and central authority, as well as the user and data owners are merged. For the goal of this work the thread model and different trust level of the entities are defined as:

\begin{itemize}
	\item The \textbf{Central Authority} is assumed to try to break into the client communication. It might provide false information or wrong states to trick any identity to provide sensitive content. However, the central authority will cooperate and will only deviate from the protocol if some advantage could be gained. Simply denying the cooperation - and so disrupting the service - does not belong to the goals of the central service. The central service does not cooperate with other malicious entities. 
	\item Each \textbf{Attribute Authority}’s goal is to access and eve drop content that is encrypted with attributes outside if its domain. This distrust can be leveraged by a trust relation ship, where two attribute authorities can explicitly agree to trust each other. This mutual trust relation indicates that an trusted attribute authority is assumed to provide only truthfully information to the other trusted authority. Again it is assumed that an attribute authority does not cooperate with other malicious entities and that it will follow the protocol.
	\item \textbf{Cloud Storage Provider} are simply assumed to be honest-but-curious. They follow the protocol by try to decrypt content if possible. A cloud storage provider is separated from the other system and does not cooperate with any other malicious entity.  
	\item \textbf{Users} are untrusted. They try to collude with each other to achieve a higher level of decryption power. That means that they will exchange private attribute and two factor keys (see Section \ref{sec:dac-macs}). However, it is assumed that they will not sell a decryption black box so that other external, unauthorized or revoked users use the black box to access sensitive content. 
\end{itemize}

This thread model was adapted from that in TF-DAC-MACS \cite{li2017two}. Here the central authority and certificate authority are also semi-trusted. This is compensated by leveraging the trust level of the attribute authorities by introducing the trust-relationship. In TF-DAC-MACS a lot of trust is rooted in the certificate which was issued by the trusted certificate authority. In a real life system the trust of the certificate authority needs to be deescalated, since it is usually controlled by the same entity as the central authority or at least act on its behave. To do so, for each certificate validation a second channel needs to be established: One to the issuer and one to the subject of the certificate, asking both for the validity of the certificate. 

\section{Requirements}
\label{sec:requirements}
Based on the previous introduced entities in ABE \ref{sec:attribute-based-encryption}, the thread model \ref{sec:thread-model} and the security requirements of a secure-cloud storage system the following requirements can be extracted.

The general (\req{B}) requirements of this thesis will be summarized by two major points: 

\begin{itemize}
	\item[\req{B1}] \textbf{Performance:} Participating in the system should be possible with low-performance devices (such as smartphones). The overhead for the server on proxy decryption, attribute issuing and revocation should be reasonably low.  
	\item[\req{B2}] \textbf{Scalability:} The system should scale better than the current encryption scheme with respect to the number of file keys. This summarizes the performance impact of a growing number of clients.
\end{itemize}

In addition, the core (\textbf{\textit{C*}}) security requirements in the context of an \ac{MA-ABE} scheme are the following:
\begin{itemize}
\item[\req{C1}] \textbf{Collusion resistance:} For two users it should not be possible to combine their attributes to achieve a higher level of decryption power. Collusion resistance need also be ensured also on revocation. 
\item[\req{C2}] \textbf{Inter-Company Sharing:} Each company is only responsible for its own domain. This includes attribute and user administration, which translates to secret key generation and revocation. 
\item[\req{C3}] \textbf{Central Authority:} The Central Authority (\ac{CA}) shall not have global decryption power. At most an Attribute Authority (\ac{AA}) can decrypt user files of its own domain.  
\item[\req{C4}] \textbf{Secret Master key (if any):} Key recovery requires a secret and securely stored master key. It should solely function in the company domain and not globally. 
\item[\req{C5}] \textbf{Large Attribute and Key Universe:} The number of attributes and users shall not be restricted.
\item[\req{C6}] \textbf{Adding new Attribute Authorities:} It should be possible to add new AAs at runtime. Without either shutting down the system or recreating each key.
\item[\req{C7}] \textbf{Untrusted Attribute Authority:} A corrupt AA can only harm its own domain, but can not harm the outside system in any way. It cannot gain any additional information.
\item[\req{C8}] \textbf{Key and Attribute revocation:} Revocation is needed to handle user management in terms of attribute promotion, attribute demotion and key revocation. Forward secrecy should be provided.
\end{itemize}

\noindent Other (optional \textbf{\textit{O*}}) requirements are: 
\begin{itemize}
	\item[\req{O1}] \textbf{Traitor tracing:} A user in \ac{ABE} is described by his attribute set and is anonymous in this set. Misbehaving users, who sell their attribute keys to create a decryption black box, should be identifyable \cite{liu2016practical}.
	\item[\req{O2}] \textbf{Fine-grained access control:} The user shall not be bounded on defining fine-grained access policies which requires either an access tree \cite{bethencourt2007ciphertext} or an linear secret sharing scheme (\ac{LSSS}) \cite{yang2013dac} \cite{li2016secure} \cite{wu2017security} \cite{li2013matrix} \cite{liu2016practical}.
	%Some schemes restrict the user to threshhold access policies, where a user needs at least $n$ of $m$ attributes to encrypt the cipher text. \cite{chase2007multi} \cite{chase2009improving} Other approaches are restriced to $\ac{AND}$ gates which would translate to $m$ of $m$ threshhold gates. \cite{li2017two}
\end{itemize}

\section{Contribution}
In this work different schemes suitable to resolve the rekeying problem will be compared. First, the group communication schemes on a theoretical bases are compared. Then it is argued why ABE might scale even better and different schemes are compared on a practical level. To do so a homogeneous platform was used and different benchmarks will be performed to evaluate the performance and scalability of thous schemes. Thous benchmarks are novel in that matter that no other research compared the different ABE topics on a practical level.  

In the second part of this work, an prototype implementation of the selected approach will be given which will stay as close as possible to the current scheme in Bdrive. To make the selected scheme practically applicable and to fit the requirements of section \ref{sec:requirements}, small adaptation to the scheme will be made. The target will be to compare the current scheme and the ABE approach. Further, a conclusion of the applicability of ABE in the real world is drawn and it is evaluated whether ABE does also scale better in real life scenarios. 

% Concept
% Secure group communication
\section{Secure Group Communication}
The most sophisticated approach to the scalability problem is to create a secret \textit{group key} (\ac{GK}) each time multiple devices need to access the same file. This schemes are called "Secure Group Communication" (\ac{SGC}). The main idea is to reduce the number of filekeys by encrypting all files with a unique \ac{GK} that is only known to the members in the share. Simple approaches use a so called \textit{Key Distribution Center} (\ac{KDC}) to distribute the group keys to each member. However, while Bdrive could manage forward secrecy quite easily, this schemes suffer from additional reencryption overhead if a member leaves the group. 

% Mutlicast in group communications
Commonoly Secure Group Communications are mentioned together with \textit{multicast messages}. With multicast the same message is distributed to different devices in one setting. In contrast stands \textit{unicast}. Here a central authority is required that manages authentication and authorization to decide which entity should access which content. With multicast messages, however, content is distributed to each entity regardless if this entity is authorized to access this content. Instead authorization is handled via encryption of the content. If an entity owns the key to decrypt the data it is implicitly authorized to access the content as well. Multicast messages are especially handy in group communications since they only need to send one message to distribute the information to all members. As a the trade-off the message size increases.

% The role of \ac{PKI} in group communications
While there are many approaches researching secure group communication and making the rekeying process more efficient, the number of schemes can be reduced to those applicable to secure could storage. Here some prerequisites are given that are not generally assumed. Secure cloud storage system often deploy a public key infrastructure (\ac{PKI}) to bind identities to public keys of clients which means that a key exchange already happened.

In contrast to that are schemes that extend the 2-parties Diffie-Hellman (\ac{DH}) key exchange to n-party scheme, such as in \textit{Steiner at. al., 1996}\cite{steiner1996diffie}. The \ac{DH} key exchange is designed for unsecured environments to create a unique communication key, known to every member in the group, but not known to a passive attacker. However, since all communication is already disclosed and secured by the \ac{PKI} those schemes can be eliminated. They suffer from additional computation or message exchange overhead compared to schemes that assume an existing \ac{PKI}.   

In the following it is widely assumed that a client is identified uniquely by its public key, which is also known by the cloud server and can be queried by other clients. 

\subsection{Group Key Management Protocol}
The Group Key Management Protocol (\ac{GKMP})\cite{harney1997group} addresses the rekeying problem in a simple but scalable way. If a device want to share files with a group of other clients it creates the group key and encrypts all filekeys with this group key. This \ac{GK} needs to be securely distributed to all members which means that the data owner needs to download and store every public key of the group members and encrypt the \ac{GK} for each of them. Further, the data owner needs to encrypt all shared filekeys symmetrically with the chosen group key. 

% Disadventage
To ensure forward secrecy, the \ac{GK} needs to be renewed every time a member leaves the group. This overhead is located in the same computational magnitude as bootstrapping a new group. 

% Adventage
The big advantage of this scheme, compared to the scheme employed in Bdrive, is that it reduces the overhead of the number of encryptions, messages and keys on new file upload dramatically. While the data owner had to create a filekey for each member respectively, \ac{GKMP} reduces this to one filekey per upload. Additionally, since no backward secrecy have to be ensured, the \ac{GK} only have to be distributed to the new member\footnote{With backward secrecy the GK would have been renewed each time a new member joins.}.

\subsection{Logical Key Hierarchy}
\begin{figure*}[!ht]
\centering
    \includegraphics[width=0.8\linewidth]{img/LKH.png}
    \caption{Balanced binary \ac{LKH} and user H leaves the group. Node marked in red is removed from the group. Node marked in yellow require an update and need to be stored locally by any leaf of this node. }
    \label{fig:lkh}
\end{figure*}

Logical Key Hierarchy (\ac{LKH})\cite{wallner1999key} is a \ac{GKMP} that reduces user side storage and rekey transmissions by organizing users keys in a tree hierarchy and maximizing the use of multicast messages. The tree is maintained by a central \textit{Key Distribution Center} (\ac{KDC}). In the work of Wallner \textit{et. al.} \cite{wallner1999key} it is explicitly stated that this trees neither have to be balanced nor binary, but for the sake of simplicity exactly this is assumed. Users (and their respective public keys) are organized at the leafs of the LKH. Each node is composed of a key that is encrypted with its children keys. An encrypted key is known to all leafs related to this node and can, on change of this node, be transmitted in one multicast transmission to all its leafs. The root node summarizes the \ac{GK}. The \ac{GK} is used in the same way as in \ac{GKMP} to secure the group communication. 

While in \ac{GKMP} each user needs to store each members public key to eventually rekey the GK, in \ac{LKH} each user only needs to store $log(n) +1$ keys (path from leaf to root node). Beginning from the bottom, each leaf node could decrypt the next parent node until it arrives at the root node. One other advantage is that many members share the same keys. So it is more efficient to transmit this keys in a multicast setting. \ac{LKH} needs to do $2 log(n)$ multicast transmissions and analogous $2 log(n)$ encryptions on each member leave or join.  Here each updated key in the path from leaf to root node needs to be updated and encrypted two times: One time with the key of right children node and second with the key of the left children node. 

In figure \ref{fig:lkh} user H recently left the group which means, that to ensure forward secrecy, the keys in node $GH$, $EG$ and $GK$ must be updated. The key server will choose a new $GH'$, $EG'$ and $GK'$, encrypts and distributes them in a bottom up fashion starting by $GH'$. $GH'$ is encrypted with user Gs and user Hs public key (G-PK, H-PK) and transmitted to them respective. Next, user E and F receive in a multicast transmission $EG'$, which is encrypted with $EF$ and user G and H receive in a multicast transmission $EG'$ which is now encrypted with the newly established $GH'$ key. Finally, the GK needs to be updated. For the left sub tree $GK'$ is encrypted with $AC$ and transmitted using multicast to the users A, B, C and D. And for the right sub tree $GK'$ is encrypted with $EG'$ to transmitted using multicast to user E, F, G and H. Adding a user to the group follows the same principle. Backward secrecy is implicitly given and not easily removable from this scheme.

\subsection{One-Way Function Trees}
To reduce the transmission and encryption overhead of $2 log(n)$ even further to $log(n)$ other schemes use pseudorandom functions \cite{canetti1999multicast} or one-way functions \cite{sherman2003key} to compute the required keys in the path. This schemes are strongly related to \textit{Merkle Trees}, since each update of the user set will force an update of the root node: The group key.

Each user stores a blinded key for each sibling node in the path to the root node. Starting from his individual key, a user can compute the blinded version of his key. To compute the next parent key, a user utilizes his blinded key, together with the siblings blinded key, which is fed into a one way function to create the key for the parent node. This node needs to be blinded to serve as the input for the next level. In such way the user can compute the needed keys up to the GK. 

Lets assume the same tree layout as in figure \ref{fig:lkh} but assume that user H joined the group. Given a cryptographically secure hash function $h$. First, user H needs to receive the blinded keys, $h(AC)$, $h(EF)$ and $h(G-PK)$ which he stores locally in addition to his own public key $H-PK$. \textit{One-Way Function Trees} (\ac{OWFT}) have the same storage overhead as LKH with $log(n) + 1$.  As already known from LKH, $GH$, $EG$ and $GK$ need to updated. This nodes are computed from their children blinded values using a merging function $m$ which takes two inputs and produces one output. This could for example be the concatenation of both inputs and then applying $h$ to receive a fixed length output value. Starting from the bottom,  $GH'$ is computed as $m(h(G-PK), h(G-PK))$, analogous $EG' = m(h(EF), h(GH'))$ and finally the updated group key $GK' = m(h(AG), h(EG')$. The new blinded keys are distributed to their respective users using one multicast transmission each, reducing the transmission and encryption overhead to $log(n)$. If user H leaves the group, user G must come up with a new random secret that he hashes as placeholder for the previous blinded key of H to compute $GH'$. Thus forward and backward secrecy is enforced.


\subsection{Comparison}  

\begin{table*}[!ht]
\centering
\begin{tabular}{l 		| l 						| l 							| l 						| l 						}
 						& \textbf{Bdrive} & \textbf{\ac{GKMP}}\cite{harney1997group} & \textbf{\ac{LKH}}\cite{wallner1999key} & \textbf{\ac{OWFT}}\cite{sherman2003key} \\%\textbf{\ac{GDH}.1}\cite{steiner1996diffie}\\
\hline
\textbf{inizial share} 																																		\\
keys 					& $nf$	 					& $1$  							& $n-1$  					& $n$	 					\\%& $1$ 			\\
messages (unicast)		& $nf$	  					& $n$ 							& $n(log_2(n) + 1)$ 		& $2n(log_2(n) + 1)$		\\%& $2(n - 1)$	\\
messages (multicast) 	& $nf$	 					& $n$ 							& $log_2(n) + 1$ 			& $n - 2$ 					\\%& $2(n - 1)$ 	\\
encryptions				& $nf$	 					& $f + n$ 						& $f + n -1$				& $f + n -1$				\\%& $f + n$		\\
\hline
\textbf{member join} 																																		\\
keys 					& $f$   					& $1$  							& $3 log_2(n)$				& $log_2(n)$				\\ %& $1$			\\
messages (unicast)		& $f$  						& $1$			 				& $2(n - 1)$				& $n$  						\\ %& $2(n - 1)$	\\
messages (multicast) 	& $f$ 	 					& $1$ 		 					& $2 log_2(n)$				& $log_2(n)$				\\ %& $2(n - 1)$	\\
encryptions				& $f$  						& $f + 1$		 				& $f + 2(log_2(n))$ 		& $f + log_2(n)$			\\ %& $f + 2$	 	\\
\hline
\textbf{member leave}																																		\\
keys 					& $0$						& $1$			  				& $3 log_2(n)$				& $log_2(n)$				\\ %& $1$			\\
messages (unicast)		& $0$						& $n$			 				& $2(n - 1)$ 				& $0$	  					\\ %& $2(n-1)$		\\
messages (multicast)	& $0$						& $n$			 				& $2 log_2(n)$				& $log_2(n)$				\\ %& $2(n-1)$		\\ 
encryptions 			& $0$						& $f + n$ 						& $f + 2 (log_2(n))$ 		& $f + log_2(n)$	 		\\ %& $f+n$			\\
\hline	
\textbf{addition of filekey}																																\\
keys 					& $n$		 				& $0$							& $0$	 					& $0$		 				\\ %& $0$			\\
messages (unicast)		& $n$		 				& $0$	 						& $0$ 						& $0$		 				\\ %& $0$			\\
messages (multicast)	& $n$ 						& $0$ 							& $0$ 						& $0$	 					\\ %& $0$			\\
encryptions				& $n$ 						& $1$ 							& $1$ 						& $1$		 				\\ %& $1$			\\
\hline
\end{tabular}
\caption{Comparison of secure group communication schemes. $n$ donates the number of clients and $f$ the number of file keys. Forward secrecy is ensured in each scheme. Assuming a balanced binary tree as in figure \ref{fig:lkh}}
\label{tab:comparisons}
\end{table*}

Table \ref{tab:comparisons} will compare the discussed schemes against each other regarding number of keys, messages, and encryptions on initialization of the group, when a member joins or leaves and on addition of a new file. 

It is assumed that clients already downloaded the public keys of their group members. Forward secrecy need to be ensured every time a member leaves the group.
Further, messages in table \ref{tab:comparisons} are transmissions that contain a key. There are also meta messages that notify the members about a new file upload or the removal of a member, etc.  They are ignored since they add a constant overhead to all schemes.  If a message could be processed by multiple communication partners it can be transmitted in multicast transmission. In general it is assumped that $f > n$ holds true, since there are usually more files in a share than devices.

Reviewing the performance comparison, Bdrives scheme is not optimal compared to the different approaches. In fact Bdrive has the worst performance on initialization, addition of a filekey and on average also on member join. However, Bdrive has great performance on member leave since client simply do not have to encrypt anymore for the left member.

Each of the schemes have their own strength and weaknesses. \ac{GKMP} has the smallest initialization overhead and \ac{OWFT} the best rekeying overhead. As some could clearly extract from the table that all secure group communication schemes perform better than Bdrive on file upload. Bdrive needs to distribute and encrypt a new filekey $n$ times for each device while any scheme using a \ac{GK} only needs one encryption and one multi cast transmission. 

Under the assumption that a secure cloud storage does not need backward secret, \ac{GKMP} is selected as most suitable candidate. It has great initialization overhead is easily understandable and profits from the fact that no backward secrecy on member join is needed. \ac{LKH} and \ac{OWFT} both have backward secrecy fixed implemented resulting in a bit more worse performance. 

% solution: ABE
\subsection{An introduction to the field of Attribute-Based Encryption}
A key requirement in security of distributed systems is authenticity. It describes the principle of associating individuals with their digital representation. A so called \textit{public key infrastructure} (\ac{PKI}) uses certificates the validate the public key bound to an identity.

Identity-based encryption (\ac{IBE}) tries to circumvent this standard by binding identities to unique strings already associated with this identity, such as an email address. If we could encrypt directly with this identifier, we would not need a certificate check to validate the identity of the recipient. 

\subsubsection{Identity Based Encryption}
The idea of using universal identifier instead of public keys to identify individuals goes back until 1984. Shamir proposed the first \textit{Identity-Based Encryption} (\ac{IBE}) \cite{shamir1984identity} to use a central authority to create for each newly registered user a unique universal identifier (e.g. the email address) that can be used for encryption as we use to do with a public key. However, the difference to \ac{RSA} is that, since the universal identity is the public key, the use of a \ac{PKI} becomes nugatory. The main point was constructed around the idea of sending an email to a yet unknown coworker. Since the sender has to know the email address of the recipient he can derivative public key component from it, encrypt the email with this key and can be sure that only the identity owning this email can decipher the content. 

A central authority was introduced since the users private key are derived from a random seed only known to this authority. If this seed was publicly known each user could simply compute the private key from any public identifier and so confidentiality would be broken. 

Since this scheme could neither be shown to be practical applicable, nor to proven secure, it was not until 2001 when Boneh and Franklin proposed in \cite{boneh2001identity} a new approach to identity-based encryption. The use of the \textit{Weil pairing} revolutionized the field of identity-based encryption.\footnote{Curious reader can readup on the weil pairing here \cite{Miller2004} \cite{galbraith2008pairings} or here \url{https://medium.com/@VitalikButerin/exploring-elliptic-curve-pairings-c73c1864e627}}

\subsubsection{Bilinear Mappings}
\label{sec:bilinearmappings}
Bilinear mappings are a tool for \textit{cryptographic pairings} and define the relationship between two cyclic groups of the same order into a third one. A cyclic group $G$ is defined by an generator $g$ of that form that each $g^n \in G$ with $n \in \mathbb{Z}$ completely describes each element in the group $G$.

A bilinear map function $e$, also called \textit{pairing} is now defined as the mapping between two groups of the same order $G_1$ and $G_2$ into $G_t$:

\begin{equation}
 e : G_1 \times G_2 \rightarrow G_t \\
\text{ such that for all } g_1 \in G_1, g_2 \in G_2, a, b \in \mathbb{Z} \\
e(g_1^a, g_2^b) = e(g_1, g_2)^{ab} 
\caption{Definition: Pairing}
\end{equation}

If $G_1$ and $G_2$ describe the same group the mapping is called symmetric. In fact the \textit{decisional bilinear Diffie-Hellman problem} becomes easily computable using bilinear mappings. \cite{bethencourt2015intro}

This construct of pairings is used by many schemes to either distribute a secret or computing the secret without revealing it.

\subsubsection{Attribute-Based encryption}
Attribute-Based Encryption (\ac{ABE}) is ancestor of IBE. Instead assigning each user a unique identifier, the work of Sahai and Waters \cite{sahai2005fuzzy} describes the approach of assigning each user a descriptive set of attributes. A deeper dig into this topic revealed the well known setup from \ac{IBE}: a central \textit{attribute authority} (\ac{AA}) issues attributes on user registration, binds them to key pairs and serves as a central public-key directory. 

The advantage is that the \ac{AA} doesn't have to be online all the time. Once the attributes are issued and distributed the data owner, without any further interactions with the \ac{AA}, can use those to share content to authorized members. Further, no server needed to check whether entities are authorized to access certain content. This follows due to the fact that if the entity was authorized to download the content, it will certainly also be able to decrypt it. 

As we already know, content is encrypted under attributes, rather then users. This implies that the data owner encrypting the file does not really know or does not care which individual deciphers the ciphertext. Using Boolean formula and the given attribute universe, the data owner creates an access policy securing his ciphertext. Any entity owning a certain set of attributes is allowed to access the data. This feature comes in handy in specific domains such as the medical domain. Here any set of doctors satisfying certain attribute are able to read the medial record. The patient, as the data owner, does not care about the real identities of the doctors only if they satisfy a specific qualification goal. The same is applicable on file-sharing in the business domain. Here users often share content to roles or group of members and if an employee gets promoted he automatically gets access to the content addressed to his job level. 

Further, user management systems always also need a procedure to revoke users and their attributes. 
In general two steps need to be made to revoke an attribute from an user. All users, except the revoked user, need to receive a new private key for this attribute and in addition, each ciphertext encrypted under this revoked attribute need to be updated to ensure forward secrecy. 

In summary ABE provides us a big advance for the rekeying problem: Since new members joining a group just receive the respective attribute keys no group key needs to be distributed. Users are automatically member of a group if they satisfy the access policy. With \ac{ABE} we are able to go beyond the natural limitation of Secure Group Communication schemes which had a lower bound of one key per newly joined member.  

\subsubsection{Comparing Secure Group Communication to Attribute-Based encryption}
Comparing \ac{SGC} to \ac{ABE} is non tivial. This is due to a different encryption technique used by \ac{ABE}: paring. Pairing scales more or less with the overhead of \ac{RSA} rather then \ac{ECC} or block ciphers as stated by \textit{Galbraith et. al.} \cite{galbraith2008pairings}. But to archive the same bit security computations in pairing need to happen in a much bigger bit-field rendering \ac{ABE} les performant in comparison to \ac{RSA}. But in specific scenarios \ac{ABE} uses less keys to setup the group communication. In the following thouse secnarios will be described and analysed where this schemes differe and what use-cases \ac{ABE} schemes address.

In a \ac{RSA} sharing scheme, each user has his own public key which needs to be transmitted to a thrid party to establish a secret connection. Creating a group under this scheme will implicitly force the data owner to retrieve all $n$ public keys of all $n$ group members. The central server needs to provide a \ac{PKI} together with the public keys of each registered user to proof their identity. Thus follows the contrain that the central server needs to be available all the time to provide public keys for new registered users. Each member in the group receives an encrypted copy of the group key. The number of keys in \ac{SGC} scales at least linearly. 

\ac{ABE} breaks the contrains that the central server has to be available at all time. This is done by using attribute keys on encryption rather then the users public keys. This reduces the number of keys that need to be maintained to the size of the attribute set describing the group. If an new user is registered in the system no new keys need to be downloaded from the view of the data owner. The registered users retrieves his attribute set and eventually can decrypt the \ac{GK} if the his attributes statisfy the access poliy of the group. Further notable is that the number of keys that are maintained in the group remains constant in $a$. With $a$ donating the number size of the attribute set describing the group. 

Given this observation we can state that \ac{ABE} is adventagious on scenarios where users address an unknown group of individuals. This can be some departments (e.g. police department of New York), colloquiums (e.g. faculity for Secure network communications), or general user groups (all employees working in the security research team).

To better clarify the scalaing adventage of \ac{ABE} lets take table \ref{tab:comparisons} translate it to big $O$-notation and compare it to \ac{ABE} in general. Further we use \ac{GKMP} because it got good overall performance.

\begin{table*}[!ht]
\centering
\begin{tabular}{l 		| l 						| l 						| l }
 						& \textbf{Bdrive}			& \textbf{\ac{GKMP}} 			& \textbf{\ac{ABE}} 		\\
\hline
\textbf{inizial share} 																				\\
keys 					& $O(nf)$ 					& $O(1)$	 				& $O(1)$			\\
messages (unicast)		& $O(nf)$  					& $O(n)$					& $O(n)$			\\
messages (multicast) 	& $O(nf)$ 					& $O(n)$ 					& $O(1)$			\\
encryptions				& $O(nf)$ 					& $O(f + n)$				& $O(a)$ 			\\
\hline
\textbf{member join} 																				\\
keys 					& $O(f)$   					& $O(1)$					& $O(1)$			\\
messages (unicast)		& $O(f)$  					& $O(1)$  					& $O(1)$ 			\\
messages (multicast) 	& $O(f)$ 	 				& $O(1)$					& $O(1)$ 			\\
encryptions				& $O(f)$  					& $O(1)$					& $O(1)$ 			\\
\hline
\textbf{member leave}																				\\
keys 					& $O(0)$					& $O(1)$					& $O(1)$			\\
messages (unicast)		& $O(0)$					& $O(n)$  					& $O(N(a^{-1}))$	\\
messages (multicast)	& $O(0)$					& $O(n)$					& $O(1)$ 			\\ 
encryptions 			& $O(0)$					& $O(f + n)$ 				& $O(F(a^{-1}))$	\\
\hline	
\textbf{addition of filekey}																		\\
keys 					& $O(n)$	 				& $O(0)$					& $O(0)$			\\
messages (unicast)		& $O(n)$	 				& $O(0)$					& $O(0)$			\\
messages (multicast)	& $O(n)$ 					& $O(0)$ 					& $O(0)$			\\
encryptions				& $O(n)$ 					& $O(1)$					& $O(1)$			\\
\hline
\end{tabular}
\caption{Comparison of Bdrive, \ac{GKMP} and \ac{ABE} scheme. $n$ donating the number of members, $N$ the number of all users in the system, $f$ the number of file keys in the group, $F$ the number of all filekeys, $a$ the number of attributes used for this group, $A$ all attributes }
\label{tab:comparisonsOWFTtoABE}
\end{table*}

Note that we assume the destribution of $a < n < f$ (number of attributes in the system is smaller then the number of devices which is maller then the number of file keys) with growing number of users. While this assumption does not nesscarly hold true, on average it this constratin will be satisfied. Under this assumption we can extract from table \ref{tab:comparisonsOWFTtoABE}, that \ac{ABE} indeed scales better then \ac{OWFT} or Bdrive on inizialisation and member join. On Member leave, however, is difficult in \ac{ABE}. Most likly the member leave would describe a degregation or revoking of an attribute. \ac{ABE} sufferes from additional due to updating the attribut key for each user owning this old attribute ($N(a^{-1}$) and additionally updating all cipher text that were encrypted with the attribute ($F(a^{-1}$).

On the meta level attribute-based encryption tackelts the rekeying problem by focusing on attribute and groups rather then individuals. \ac{ABE} reduces the number of keys needed by resuing and combining exisiting keys. In constract, secure group communciation schemes need to create a new key per each group. Here \ac{ABE} exploits the fact that groups generally can be described by an unique attribute set. Implicitlyu it follows that if another group is described by the exact same set of attributes the same keys are used. So the total number of groups is limited to all possible combinations of attributes. In contract stands secure group communication where a new group is bounded to a new group key. An unlimit number of groups can be created.

%Lets define an scenario adventagous to \ac{ABE}. Alice wants to share a file with all management members of the coffee company. Since she does not know the members in person, nor their email addresses, she simply creates a share with the group "management of coffee company". Alice only needs to retrieve the key of the management from the central server of the coffee company. This proceedure took Alice, one encryption and two tranmissions: one to retrieve the key and one to upload the encrypted text. 

%If we apply the same scenario to \ac{SGC} we face a problem. How to know which public keys belong to the executive officers? Alice need to check on the webside which people are in charge of the coffee company, to download thier public keys, encrypt the group key with their public keys, and upload the file and the \ac{GK} for each manager. This took alice one lookup of the role to key mappings, $n$ downloads of public keys, $n$ encryptions of the \ac{GK} for each member and one encryption of plain text, and two uploads of the cipher text and the group key.         

In conclusion is \ac{ABE} more sutable for to make the rekeying process of Bdrive more scalabe. We can clearly see that \ac{ABE} scales with the number of attributes which is assumped to be less then the number of clients. Further, \ac{ABE} does not only handle the encryption but also provides an authentication service. Users are bound to roles and attributes which are tidly interleafed with the encryption scheme. Bdrive target audience are business which by nature have some kind of attribute authority in the form of role managment and authorization mechanism embedded. Here \ac{ABE} can enfold its true potential and outperform Secure group communication schemes not only in efficiency but also in additional security features. 



% ABE
\subsection{Basic Attribute-Based Encryption}
In the following sections we will have a look at the different ABE schemes. We will extract the  characteristics of each subtopic of ABE, select a representative candidate and finally, a practical performance comparison to evaluate the scalability is done. 

A suitable candidate in the field of ABE is a scheme that satisfies the requirements stated in section \ref{sec:requirements}. The requirements are a super set of thous defined in \cite{lee2013survey}. For the basic \ac{ABE} schemes we will focus on the requirements \req{C1}, \req{C3}, \req{C5} - \req{C8} and optional requirements \req{O1} and \req{O2}.
The general requirements \req{B1} and \req{B2} will be evaluated by the practical performance and scalability analysis.

Some requirements are left out since not all requirements are directly applicable to all ABE schemes. We need to find the comparable subset of requirements on which bases we are able to compare the schemes. The more we dig deeper into the topic the more requirements can we include in our analysis. 

\subsubsection{Collusion resistance (C2 requirement)}
Lets construct a very basic attribute based encryption scheme to clarify the importance of collusion resistance. Assume a distributed crypto-system based on \ac{RSA}. We setup an \ac{AA} which binds attributes to \ac{RSA} public-private key pairs. The attribute "student" gets bound to $K_{PR(s)}$ for the private key and $K_{PU(s)}$ for the public key. Attribute "works at TU Berlin" gets bound to the Key $K_{PK_PR(tu)}$ and $K_{PU(tu)}$ respectively. Now, we setup our very own first ABE scheme. The AA can pin the public keys of each attributes to its public billboard so that every entity in the system can use the public attribute keys for encryption. Each user who is currently a student receives a copy of the private key $K_{PR(s)}$ and each user who is currently working at TU Berlin receives a copy of $K_{PR(tu)}$. 

A user, lets call her for historical reason Alice, wants to share content with all students that are also working at TU Berlin. With our basic scheme Alice is able to use \textit{layered encryption}\footnote{Layered encryption: encrypting a plain text with multiple keys, forcing the decryptor to own all relevant keys} to create an \textit{AND}-policy for her cipher text. She encrypts the plain text $p$ with both public attribute keys $c = K_{PU(s)}(K_{PU(tu)}(P))$ and publishes the ciphertext $c$ to a public CSP so that everyone can download it. Students that are working at TU Berlin owning both private keys can decipher the ciphertext by applying both private keys in reverse order $K_{PR(tu)(K_{PR{s}}(c)}$.

The attentive reader will have notice a crucial security leak in this scheme. This \ac{ABE} scheme is not collusion resistant. In fact, collusion resistance is a core requirement of any ABE scheme. On paper it is defined as the impossibility of any two attribute holder to combine their attributes to archive a higher level of encryption. Lets assume that Bob is a student and Eve is working at \ac{TU} Berlin. Both users received their private key. Now they can simply exchange their attribute keys so that they are both able to decipher the ciphertext even if they separately do not own both attributes.  

Usually collusion resistance is ensured by issuing each user a private key that is blinded by a random value. This random value will vanish on decryption. However, if two users collude they will mix their blinded values resulting a plain text that is still blinded by some unknown value. To ensure collusion resistance pairings are commonly used where a user specific id (\ac{UID}) is generated and bound to the exponent of the user's attribute private key. The decryption method will be designed in that way that the blinding of the attribute private key will vanish. 

\subsubsection{\ac{KP-ABE}}
In Key-Policy Attribute-Based Encryption (\ac{KP-ABE}) \cite{goyal2006attribute}, is a \ac{ABE} technique that associates ciphertexts with attributes that fulfill a policy embedded in the private key assigned to a user. 

To evaluate if a policy matches given attributes an \textit{Access Tree} or \textit{Linear Secrect Sharing Matrix} (\ac{LSSS}) is used. While both representations represent the same Boolean formula a \ac{LSSS} is often more efficient. For a better explanation the model of an access tree will be used. For each node of this tree it is evaluated whether the children satisfy a certain condition. This might be implemented as \textit{OR} or \textit{AND} threshold gates. While \textit{OR} can be expressed as $1$-out-of-$n$ children need to satisfy the condition, \textit{AND} conditions are $n$-out-of-$n$ threshold gates meaning all children need to satisfy the condition. This approach is called a monotonic access structure and often referrers to \textit{Threshold Security} or \textit{Threshold Access Structure}. 

The initial paper \cite{goyal2006attribute} did leak some basic requirements that was improved by other work. Such as a fix-length cipher text and attribute revocation. Further, the authors stated that users, once issued a policy, are able to delegate a subset of their access tree to other users. Users them self could act as a new attribute authority to issue other uses a subset of their decryption power. While this is a nice feature for some use cases, in a cloud system the provider would like to restrict this delegation or would like to be able to track traitors selling their private keys for black box decryption.

KP-ABE is meant as a broadcasting encryption scheme used by radio providers so that they can address any user that bought a paid membership to distribute premium content. However, since the policy was bound to the users key it is not possible to address a ciphertext to a specific user group without knowing each users access policy. Only the central authority, which issued the users private keys in the first place, would know who could decipher the encrypted text. This in fact renders \ac{KP-ABE} impractical for the application in a cloud sharing scheme. User would simply don't know who are able to decrypt the cipher text. 

The certainly very specific use case of \ac{KP-ABE} probably led to the circumstances that not much paper using this technique exists. So few that it was hard to find any paper satisfying all requirements previously specified. The only paper that was found implemented direct revocation using negation of attributes \cite{lewko2010revocation}. This would in turn lead to the fact that users private key sizes would grow with each revoked attribute. 

\subsubsection{\ac{CP-ABE}}
In contrast to KP-ABE stands Cipher-text policy Attribute-Based Encryption (\ac{CP-ABE}) \cite{bethencourt2007ciphertext}. \ac{CP-ABE} assigns ciphertexts with a policy and user keys with attributes. This enables \ac{CP-ABE} to be much more flexible and verbose in comparison to \ac{KP-ABE}. The big advantage is (similar to \ac{KP-ABE}) that the need for a central authentication server is obsolete. Each ciphertext could simply be accessed by every user and only thous who have the right attribute set present are able to decipher the ciphertext. 

The lack of revocation \req{C8} in \cite{bethencourt2007ciphertext} and the need for a large attribute and user universe was huge \req{C5}. Revocation means that a system manager could revoke attributes from users or even the user himself - in modern company settings an required feature to ensure forward secrecy. Large attribute universe, on the other hand, describes that the number of attributes that can be distributed by the \ac{AA} is so large that it is unlikely that a system will run out of attributes to issue. 

The first proposal of a revocation scheme in \ac{CP-ABE} was in 2010 \cite{liang2010ciphertext}. This scheme used a similar approach like \ac{LKH} to make the revocation process efficient. In 2016 Lui \textit{et. al} \cite{liu2016practical} proposed a revocable \ac{ABE} scheme that supports traitor tracing and large attribute universe. To compare CP-ABE with the other schemes we implemented \cite{liu2016practical}. 

\subsubsection{Non-monotonic access structure of \ac{ABE}}
While monotonic access structures can describe \textit{AND} or \textit{OR} gates, non-monotonic access structure can also reference \textit{NOT} gates. This adds another layer of find-granularity into the system. In non-monotonic structures, first introduced by Ostrovsky \textit{et. al.} \cite{Ostrovsky:2007:AEN:1315245.1315270}, a user may be excluded from certain topics. 

Take for example Alice who is an intern in the Top Secret company. In monotonic access structure she would receive both attribute private keys: "Intern" and "working in Top Secret Company". By nature of monotonic access structures, she would be able to decipher all content addressed to all employees at the Super Secret company. However, some content might be so confidential that interns shall not be able to access them. This exclusion would not be possible. With non-monotonic access structure an administrator may want to encrypt certain information with the policy "working in Top secret Company AND NOT intern".  

While being an imported field in the Boolean formula and fine-grand access control domain, non-monotonic access structures did not gain the same attention as \ac{CP-ABE} did. As an candidate which shall represent this \ac{ABE} topic we selected \cite{10.1007/978-3-642-54631-0_16}. With negation of attributes revocation becomes more or less trivial. Each attribute can be versioned and simply excluded on encryption. While this approach is not scalable over time it is simple enough to be implemented in some schemes.   

\subsubsection{Multi-Authority Attribute Based Encryption}
Ontop of the previos mentioned schemes emerged a new sub topic of \ac{ABE}: \textit{Multi-Authority Attribute Based Encryption} (\ac{MA-ABE}). The main motivation was that a single \ac{AA} is not practically applicable. In the real world we face different domain each maintaining it own attributes. 

\ac{MA-ABE} while beeing a roughly young field of \ac{ABE} encryption technique enjoyed a lot of attention in the research area. In addition to the normal requirements like collusion resistance and revocation mechanism, \ac{MA-ABE} also deals with the question on how to deescalate the global decryption power of the central authority (\ac{CA}). 

A setup of a \ac{MA-ABE} system looks quite similar over the field of related work. On system inizialisation we setup the \ac{CA}. The purpose of the \ac{CA} is to bootrap new Attribute Authoritieis (\ac{AA}) which then will administer their domain. In this domain users and attributes of the \ac{AA} are located. After the \ac{CA} bootstraped all \ac{AA}s and all users are registed it could in theority go offline. 

To ensure system wide collusion resistance each user usally gets a unique user identifiery (\ac{UID}) assigned. This \ac{UID} is issued by the \ac{CA} which is the only entity having an overview about the whole system. 

A suitable condidate to compare \ac{MA-ABE} to the previous introduced schemes is \textit{hirachical attribute based encryption} (\ac{HABE}, section \ref{sec:HABE}). In short \ac{HABE} is structured around the idea of attribute administration delegation. The strcture can be imaged like the domain name system. On top level there is the root master administating the whole domain. He can delegate power in form of attribute issuing und user setup to sub entities. This sub entities can again forward a subset of their power to their childen. For a complete explenation see \ref{sec:HABE}. We will use \cite{Wang:2010:HAE:1866307.1866414} as representiv candidate for \ac{MA-ABE}. 

\subsubsection{Comparison}
To compare the selected representived of the previous section we choose the charm framework\footnote{\url{http://charm-crypto.io/}}. Here we can already find implementations for the sutable \ac{KP-ABE} candidate \cite{lewko2010revocation} and for non-monotonic \ac{CP-ABE} \cite{10.1007/978-3-642-54631-0_16}. The other two schemes such as \ac{CP-ABE} \cite{liu2016practical} and \cite{wang2011hierarchical} where implemented by us in a github frok of the charm repo\footnote{\url{https://github.com/Anroc/charm}}. 

\begin{table*}[!ht]
\centering
\begin{tabular}{l 					| l 				| l 						| l 				| l}
									& \textbf{LSW 08} \cite{lewko2010revocation}	& \textbf{YAHK 14} \cite{10.1007/978-3-642-54631-0_16} & \textbf{LW 14} \cite{liu2016practical} & \textbf{WLWG 11} \cite{Wang:2010:HAE:1866307.1866414} 	\\
\req{Scheme}						& \ac{KP-ABE}		& Non-Monotone \ac{CP-ABE} 	& \ac{CP-ABE} 		& Hirachical \ac{CP-ABE}		\\ 
\req{Security scheme}				& Biliniear maps 	& Binilnear maps 			& Biliniear maps 	& Biliniear maps 				\\
\req{Expression of access policy}	& \ac{LSSS}			& \ac{LSSS} matrix 			& \ac{LSSS} matrix 	& \ac{DNF} 						\\ 
\end{tabular}
\caption{Scheme description. }
\label{tab:comparison_baic_abe_overview}
\end{table*}
\begin{table*}[!ht]
\centering
\begin{tabular}{l 	| l					| l 				| l 				| l}
					& \textbf{LSW 08} \cite{lewko2010revocation}	& \textbf{YAHK 14} \cite{10.1007/978-3-642-54631-0_16} & \textbf{LW 14} \cite{liu2016practical} & \textbf{WLWG 11} \cite{Wang:2010:HAE:1866307.1866414} 	\\
\req{C1}			& Yes				& Yes 				& Yes 				& Yes 				\\
\req{C2}			& No				& No 				& No 				& Yes 				\\ 
\req{C3}			& No				& No 				& No 				& No 				\\ 
\req{C4}			& No				& No 				& No 				& No 				\\ 
\req{C5}			& Yes				& Yes 				& Yes 				& Yes 				\\ 
\req{C6}			& - 				& - 				& -					& Yes				\\
\req{C7}			& -					& - 				& - 				& No 				\\
\req{C8}			& Yes				& Yes				& Yes				& Yes				\\
\req{O1}			& No 				& No 				& Yes 				& No 				\\
\req{O2}			& Yes+ 				& Yes+				& Yes				& Yes-				\\
\end{tabular}
\caption{}
\label{tab:basic_abe_comparisons}
\end{table*}

In table \ref{tab:comparison_baic_abe_overview} the different scheme techniques are listed.  And in Table \ref{tab:basic_abe_comparisons} this schemes are compared to the requirements of section \ref{sec:requirements}. 

From table \ref{tab:basic_abe_comparisons} it is clear that no scheme fully fulfill all requirements. However, they reveal a good indicator where we should look next. WLWG 11 clearly makes the race here satisfying 6 of 10 of the requirements. 

\req{C6} and \req{C7} could not be sutisfied by LSW08, YAHK14 or LW14 since they are not designed to support multible authorities. 
\question{Should I respect \ac{MA-ABE} also here?}
Lets have a look at \req{O2}. LSW08 and YAHK14 both supports fine grant access controll by using a \ac{LSSS} access structure. Also both implement non-monotonic access structes which give them the extra plus. LW14, on the other hand, does not support negation of attributes, but support an arbirary access strcture like LSW08 and YAHK14. WLWG11, however, does only support access strcutures in \ac{DNF} form which makes this scheme somehow restriced in expressivness. 

\begin{figure*}[!ht]
\centering
    \includegraphics[width=1\linewidth]{img/basic_abe_comparisons.png}
    \caption{Performance and scalability comparison between the chosen schemes with increasing number of attributes. The policy for each new attribute $a_k$ with $a_k \in A$ was defined as $\bigwedge\limits_{a \in A}^a a$. Key generation relates to the creation of users key pairs given an accesspolicy or attribute set.}
    \label{fig:basic_abe_comparison}
\end{figure*}

Notable about the comparison in figure \ref{fig:basic_abe_comparison} is that while \ac{KP-ABE} (LSW08) was expected to have the biggest overhead in Keygeneration compared to the \ac{CP-ABE} schemes and the lowest in encyrption, since no policy has to be parsed and computed here. However, none of the both assumtions was true. It seams more like the time complexlity and runtime of the different algorithms suerly depends on the design of the scheme. 

Also remarkable is that WLWG11 has the best overall performance. But it must also be noticed that in the we did not include the attribute authoirity key generation. While the paper claimed to have an constant overhead on decryption it is impressive to see that it holds true. 

While the plaintext message remained contant it is clearly visible that the ciphertext length groth linearly with the number of attributes. 

Due the greate performance and the coverage of the most of the requirements of the hirachical attribute based encryption technique, the field of multi-authority attribute based encyrption will be evaluated in more depth. Regardless of the great scalability of the \ac{HABE} it comes with a non neglegtable disadventage. The root master and the top level domain master have both global decryption power. For each user regardless of the company affiliation the administator of the system will always be able to dechipher ciphertexts of thouse users. This would break end-to-end encryption complelty. So clearly we would favor solutions that support different attribute authorities so that each company can administer their own domain seperatly from each other, but we want also that root, while beeing able to boostrap new attribute authorities, is not able to decihper any ciphertext. 



% MA-ABE
\section{Multi-Authority Attribute-Based Encryption}
In the following sections, the different sub topics of \textit{Multi-Authority Attribute-Based Encryption} (\ac{MA-ABE}) will be described, analyzed given all the requirements and finally evaluated based on their performance and scalability. 

\subsection{Introduction Into Multi-Authority Attribute-Based Encryption}
Chase 2007 \cite{chase2007multi} was first known to introduce the a working \ac{MA-ABE} scheme. In her paper she describes the process on how to derive a multi-authority attribute-based encryption scheme from the single authority scheme. Her proposal focused on interpolation and the fact that no under defined linear equation system could definitely be solved as the main security assumption\footnote{\ac{LSSS} matrix are based on the same assumption.}.  

To ensure collusion resistance, each user is given blinded attribute private keys so that when used on decryption the plain text is still blinded with the user specific identifier. Using pairings (bilinear maps, section \ref{sec:bilinearmappings}) this identifier can be substituted and the plain text gets revealed. 

Chases scheme had two major disadvantages which were not addressed in her initial scheme. First, the \ac{CA} had global decryption power. That was due to the bootstrapping of the different AAs and users. The CA had to give the right seeds and global parameter to each AA to make sure that each decryption of a cipher text results in a blinded plain text that can only be decrypted using attribute keys of the same user. If this would not be the case users could easily collude. 

% That was due to the fact that it needed to issue each \ac{AA} a specific seed so that, on using this seed in a pseudo random generator, the \ac{CA} knew what random value would be calculated. This information is used to precompute the users secret key so that when later combined with the secret attribute keys, it resolves in the correct plain text. This was important since the CA, in comparison to the AA, was defined as the trusted authority. So the cipher text need this extra security layer.  

Chase improved her scheme 2009 in \cite{chase2009improving} to deescalate the global decryption power of the CA. Now all \ac{AA}s will do a $n$-party key exchange to agree on $n$ secret seeds from which the personal initial seed can be calculated. This results in the fact that no AA know the whole master secret but only its required secret seeds. Agreeing on a well known value the \ac{CA} was no longer required and as long $N-2$ \ac{AA}s are not colluding with each other the master secret remains secure. 

The other disadvantage, which is also still present in the updated scheme, is that no new \ac{AA} could be added after system initialization since it would trigger a new system bootstrap and distribution of the master secret. 

The lack of adding new attributes and the missing revocation scheme makes Chases scheme impractical for further evaluations but gives a good introduction the challenges of the \ac{MA-ABE} schemes.

\subsection{Hierarchical \ac{ABE}}
\label{sec:HABE}

\begin{figure*}[!ht]
\centering
    \includegraphics[width=0.7\linewidth]{img/HABE.png}
    \caption{Structure of hierarchical attribute-based encryption systems}
    \label{fig:habe}
\end{figure*}

\textit{Hierarchical Attribute-Based Encryption} (\ac{HABE}) is based of the idea of key and decryption power delegation. If a private key exist that have a certain access power, it is possible for the key holder to delegate a subset of his access power to a new instance. By nature follows a hierarchical structure where each user could administrate an own subdomain. Many use cases for cloud computing as well as cloud storage system emerged. \cite{Wang:2010:HAE:1866307.1866414}. While some works also take revocation into design a crucial requirement is still missing:  The domain master has always global decryption power. 

As displayed in figure \ref{fig:habe}, which is based on the approach in \cite{wang2011hierarchical}, the root master summarized the global decryption power of the system and can set up new domain masters (attribute authorities). They on the other hand can delegate a subset of their decryption power to a sub domain master. Each domain master can administer users and attributes.

\cite{Wang:2010:HAE:1866307.1866414} formalized this approach using \ac{CP-ABE}, which was later extended in \cite{wang2011hierarchical}. While this implementation serves with a revocation scheme it depends on an access policy in disjunctive normal form (\ac{DNF}). Any boolean formula can be transferred into DNF but sometimes this might enforce negation. Since \cite{wang2011hierarchical} does not provide any negation mechanism, this ABE scheme is limited in expressivenss.

\subsection{Decentralized Attribute-Based Encryption}
\label{sec:DABE}
In contrast to HABE stands \textit{decentralized attribute-based encryption} (\ac{DABE}) which is structured around the idea of having a peer network of authorities and user. No entity has a global view of the system and no entity is in charge of bootstrapping AAs. This ecosystem is self sustaining so that each entity can become an \ac{AA} if needed and start issuing new attributes to users. 

However, some form of centralization must still be present. The global parameters, for example , need to be known to each new entity and each user still need to get a unique global identifier assigned to prevent collusion. Also attributes need to be synchronized since the set of attribute identifier need to be non-intersecting across different domains. While not being a limitation it is rather a challenge to enable global communication and archive global synchronization across the network. Such a distributed system was first proposed by \cite{lewko2011decentralizing}. 

Revocation in a distributed setting remains an open issue. Since no central authority has a global view over the users, no authority can be in charge of revoking them. An authority could revoke its issued attributes but only in an indirect fashion since decentralized systems always have to take into account that nodes do not have to be online all the time. If an authority would go offline as soon at it received the revocation request the revocation procedure would never trigger. 

Cui and Deng showed in \cite{cui2016revocable} that a DABE system with indirect revocation could exist. Each key and ciphertext get a liveness assigned which make them valid for a certain time period. After this time period expires all keys need to be reissued by each \ac{AA}. Indirect revocation happens by simply excluding the user from reissuing certain attribute keys. This system is implemented in the comparison in \ref{sec:ma-comparison}.

However, question remains if such a system would be practically applicable in the real world since each ciphertext need to be reuploaded and each attribute keys need to be redistributed in each time period. Until now no direct revocation DABE was developed.

\subsection{Efficient Data Access Control For Multi-Authority Cloud Storage}
The most explored field in \ac{MA-ABE} is the \textit{Efficient Data Access Control for Multi-Authority Cloud Storage Systems} (\ac{DAC-MACS}) family \cite{yang2013dac}). First introduced by Yang \textit{et. al.} 2013 it describes an efficient, revocable \ac{MA-ABE} scheme based on \ac{CP-ABE} which uses proxy encryption on decryption and reencryption to make the scheme more efficient. 

Proxy de-/reencryption, used by \cite{yang2013dac}, \cite{wu2017security}, \cite{li2017two} and \cite{wang2011hierarchical}, is a technique where the a server helps the user on decryption to reduce the computationally intensive work. It is motivated by the fact that mobile devices often don't have much computing resources so the server helps the clients on decryption. The main idea is that the server does the preprocessing of the encrypted text given attribute keys by the user. Impotent to note is that the server will have no knowledge about the plain text since the preprocessed cipher text is still encrypted with the users public key.

DAC-MACS also features a large attribute universe, adding \ac{AA}s on the running system and deescalates the global decryption power of the \ac{CA}. In sort \ac{DAC-MACS} satisfy all the non-optional requirements.

In contrast to Chases scheme, DAC-MACS eliminates the need for the global decryption power of the \ac{CA} by issuing $k$ ciphertexts: One per \ac{AA}. \footnote{If the ciphertext does not require any attributes of an specific authority it does not have to create a ciphertext for this domain.} It does not require any coordination between authorities which enables to add new \ac{AA}s at runtime without recreating the user keys. This scheme also includes features for efficient revocation while it claims to maintain forward and backward secrecy.

\ac{DAC-MACS} is not collusion resistance on attribute revocation under the active attack model. The scheme \ac{NEDAC-MACS} (New-Extended \ac{DAC-MACS}) shows and solves this vulnerability \cite{wu2017security}. Recent studies introduce a more efficient, scalable and secure approaches such as \ac{MAACS} \cite{li2016secure} and \ac{TF-DAC-MACS} (Two-Factor \ac{DAC-MACS})\cite{li2017two}. 

All the \ac{DAC-MACS} schemes are structured in roughly the same way. They usually describe six different entities:

\begin{enumerate}
	\item \textbf{Certificate/Central Authority (\ac{CA})} The purpose of the \ac{CA} is to issue user their global identifier (\ac{GID}). Further, it bootstraps the different \ac{AA}s. The \ac{CA} remains trusted but do not have any decryption power in the system. 
	\item \textbf{Attribute Authority (\ac{AA})} An attribute authority administers its own domain. Here it issues attributes and their respective private key to the user. They only accept a user if his \ac{GID} is signed by the \ac{CA}. 
	On revocation the AA will need to update the users secret keys as well as the ciphertext encrypted with the revoked attribute key. \ac{AA}s are assumed to be honest-but-curious.
	\item \textbf{Server} The purpose of the server is to help the user with proxy re- and decryption. If an \ac{AA} broadcasts a revocation of an attribute, the server downloads all related ciphertexts from the \ac{CSP} to update them with the new attribute. 
	Further, the user may give the server his attribute private keys so that the server can precompute the ciphertext. The thread model for the server is honest-but-curious. Please note, that the \ac{CA} and the server are two separated entities that do not cooperate.
	\item \textbf{Data Owner} The data owner are users who want to encrypt content with a specific access policy. To do so they use the public available public attribute keys pinned on the bulletin board of the respective \ac{AA}. Data owner do not have to know anything about the receiving user or user groups in the system. After encryption they update the encrypted content to the \ac{CSP}.
	\item \textbf{Cloud Storage Provider (\ac{CSP})} The cloud storage provider are assumed to be untrusted but they still follow the protocol. That’s why they only receive encrypted data. They only purpose is to store the ciphertext and make them permanentally available. No authentication checks are needed.
	\item \textbf{Users} Users exist in two groups: Revoked and non-revoked. Non-revoked users try to collude with each other to get a higher level of decryption power. They download the files of the \ac{CSP} and try to decrypt them. Only if they attribute set matches the policy of the ciphertext they will be able to decrypt the file. 
	Revoked users, on the other hand, try to still decipher ciphertext. In some cases they try to collude with non-revoked user to intercept the key update key to restore their decryption rights. 
	User are in general untrusted.
\end{enumerate} 

For the comparison we will use the charm implementation of DAC-MACS \cite{yang2013dac}.

\ac{TF-DAC-MACS} counts as the most advanced \ac{DAC-MACS} scheme providing non global decryption power, secure revocation channels and both backward and forward secrecy. In addition \ac{TF-DAC-MACS} introduces the two-factor authentication. Data owner can issue and revoke \textit{authentication keys} to and from other users. This adds an additional layer of security. In total \ac{TF-DAC-MACS} archives still better performance then the other \ac{DAC-MACS} schemes providing constant decryption and encryption overhead. 
To compare the TF-DAC-MACS with the others we will implement TF-DAC-MACS from scratch. 

\subsection{Comparison}
\label{sec:ma-comparison}
\begin{table*}[!ht]
\centering
\begin{tabular}{l 					| l 									| l 									| l 					| l}
									& \thead{LTXWC 16\\(TF-DAC-MACS)\cite{li2017two}} & \thead{YJ 14\\(DAC-MACS)\cite{yang2013dac}} & \thead{LW 14\\ (HABE)\cite{wang2011hierarchical}}	& \thead{CD 16\\(DABE)\cite{cui2016revocable}} \\
\hline
\thead{Scheme}						& \makecell{CP (DAC-MACS \\ without proxy \\ 
									  decryption, 
									  with \\ two-factor \\ authentication)} & \makecell{CP (DAC-MACS \\ 
									  										  with proxy \\ decryption)} 			& CP (Hierarchical) 		& CP (Decentralized)		\\ 
\hline
\thead{Revocation}					& Direct 								& Direct 								& Direct 				& Indirect					\\
\hline
\thead{Security scheme}				& Bilinear maps 						& Bilinear maps 						& Bilinear maps 		& Bilinear maps 			\\
\hline
\thead{Expression of \\ access policy} & n-of-n threshold					& LSSS		 							& DNF 					& LSSS matrix 				\\ 
\end{tabular}
\caption{Scheme description. }
\label{tab:comparison_ma_abe_overview}
\end{table*}
\begin{table*}[!ht]
\centering
\begin{tabular}{l 	| l										| l 									| l 					| l}
					& \thead{LTXWC 16\\(TF-DAC-MACS)\cite{li2017two}} & \thead{YJ 14\\(DAC-MACS)\cite{yang2013dac}} & \thead{LW 14\\ (HABE)\cite{wang2011hierarchical}}	& \thead{CD 16\\(DABE)}\cite{cui2016revocable} \\
\req{C1}			& Yes									& No 									& Yes 					& Yes 						\\
\req{C2}			& Yes									& Yes 									& Yes 					& Yes 						\\ 
\req{C3}			& Yes									& Yes 									& No 					& Yes 						\\ 
\req{C4}			& Yes									& Yes 									& No 					& Yes 						\\ 
\req{C5}			& Yes									& Yes 									& Yes 					& Yes 						\\ 
\req{C6}			& Yes 									& Yes 									& Yes					& Yes						\\
\req{C7}			& Yes									& Yes 									& No 					& Yes 						\\
\req{C8}			& Yes									& Yes									& Yes					& Yes-						\\
\req{O1}			& No 									& No 									& No 					& No 						\\
\req{O2}			& No 									& Yes									& Yes					& Yes						\\
\end{tabular}
\caption{Requirements comparison of the implemented schemes}
\label{tab:ma_abe_comparisons}
\end{table*}


\begin{figure*}[!ht]
\centering
    \includegraphics[width=1\linewidth]{img/maabe_comparisons.png}
    \caption{Performance and scalability comparison}
    \label{fig:maabe_comparison}
\end{figure*}

To compare the sub topics of \ac{MA-ABE} we need to extend the charm framework with an implementation of \ac{DABE}, \ac{HABE} and \ac{TF-DAC-MACS}. \ac{DAC-MACS} on the other hand existed in the framework already. 

In general we extracted five steps that each \ac{MA-ABE} scheme need to provide in some form. 
\begin{enumerate}
	\item \textbf{Setup:} The global setup phase where public parameter are determined.
	\item \textbf{Authority Setup:} \ac{AA}s can register itself to the central authority (if any) and compute their secret keys. Usually, the attribute secret keys are generated. 
	\item \textbf{Register User:} In the \ac{DAC-MACS} schemes user receive public and private key components. Other schemes just assign the user a \ac{GID}.
	\item \textbf{Key generation:} Attributes are assigned to user and the respective secret keys are generated.
	\item \textbf{Encrypt:} The cipher text is encrypted by the data owner under an access policy.
	\item \textbf{Decrypt:} The user (with the help of the server) decrypts the cipher text and recovers the secret message. 
\end{enumerate}

This steps are shown and compared in \ref{fig:maabe_comparison}. We can argue that \ac{DAC-MACS} performed the worst, then \ac{DABE}, then \ac{TF-DAC-MACS} and the best performance has \ac{HABE}. \todo{extend analysis} 

However, for piratical usage especially the encryption and decryption performance is important. Here only \ac{TF-DAC-MACS} has an constant overhead while all other schemes show a linear overhead. If we further have a look at the table of requirements \ref{tab:ma_abe_comparisons}, we see that only two schemes satisfy all of the requirements: \ac{TF-DAC-MACS} and \ac{DABE}. \ac{DABE} profits also from the fact that it has a more fine grant access control than \ac{TF-DAC-MACS}. However, its indirect revocation scheme is the disadvantage that leave us with \ac{TF-DAC-MACS} as our final candidate. As mentioned in section \ref{sec:DABE} the implemented scheme uses indict time-based revocation, which forces data owner to periodically re-encrypt and re-upload their content. Since this is not shown in the comparison it would render DABE practically not applicable.

The comparison of \ac{TF-DAC-MACS} with other schemes of the \ac{DAC-MACS} family was left out in this work, since it was already done in the \ac{TF-DAC-MACS} paper \cite{li2017two}. Here some could easily see that \ac{TF-DAC-MACS} is currently the most performant scheme in the \ac{DAC-MACS} family.  

% Implementation
\section{Implementation}
As evaluated in the last section we will proceed to implement TF-DAC-MACS with small adaptions for a practical secure cloud storage system. To do so we need to implement five entities:  

\begin{itemize}
	\item \textbf{The Server} will do the inital setup of the public paramter used in the later process. It will publish this information on a public bulliton board. Further, it will trigger and permit AA creations. The server will create a \textit{Authority Identifier} (\ac{AID}) that is unique in this system. The server is assumed to be honest-but-curious.
	\item \textbf{Certificate Authority (\ac{CA})} issues a certificate for the GID of the user. This certificate can be revoced so that the user is not able to optain new secret keys from the AA oder other data owners. The CA is fully trusted.
	\item \textbf{Attribute Authority (\ac{AA})} creates the secret keys for its attributes. Attributes are prefixed with the \ac{AID} to ensure uniqueness among the attribute universe. AA are also trusted but never collude with users.
	\item \textbf{Data owner} Data owner can issue two-factor keys to trusted user. Only users owning a secret authentication key can decipher the a given cipher text if it is secured with the previous exchanged two-factor key (\ac{2FA}-Key). User are untrusted.
	\item \textbf{Cloud storage provider (\ac{CSP})} The cloud storage provider provide storage to save the encrypted files.
	\item \textbf{Users} Users download and decipher ciphertext. They receive attribute secret keys from the AA, GID and certifidates from the server and two factor keys from the data owners. The CSP is honest-but-curious as well.
\end{itemize}

In the following sections, the differene phases Setup, encrypt, decrypt, attribute revocation and authentication key revocation are shortly explained.

For en- and decryption we will still use the process of encyrpting the file symetrically to create a file key which will then encrypted under the attribute policy. This reduces the size of the content that will be encrypted with ABE to a minimum. Moreover, we can still benefit from the great performance of \ac{AES}. Please note that for any scheme details we referre to the paper of TF-DAC-MACS \cite{li2017two}.

\subsection{Setup}
\begin{figure}[!ht]
\centering
    \includegraphics[width=\linewidth]{img/TF-DAC-MACS-overview-setup.png}
    \caption{Setup phase}
    \label{fig:tfdacmacs-setup}
\end{figure}

The setup summarizes the steps of \cite{li2017two} setup, user registration, data owner registration, authority setup, keygen and authentication requests. 

The first step is to create the global public paramters on the server. Thouse paramter are exposed on a public bulliton board and queriable for each entity in this eco system. Since they are required in every step it is assumed that the entity already downloaded this informaiton. 

On AA setup the server frist generates a new AID. This AID will be mapped to the domain name of this entity. So for example the TU-Berlin will have as its ID: "aa.tu-berlin.de". In this way it is ensured that no AID can double. Further, it would be verifable via the certificate chain of TSL/SSL that this AID indeed belongs to the TU-Berlin. To do so the certificate coupled with this domain can be verified and then a query to "aa.tu-berlin.de" would lead to the corresponding AA of the TU-berlin.

The AA is free to generate attribute keys. Attributes also have identifier and values assigned to them. Values and attribute identifier can be any valid string as long as it does not contain any special characters such as ":" or ".". The attribute will have the form of "<AID>.attr.<Attribute\_name>:<Attribute\_value>". For example the major computer sience would be displayed as "aa.tu-berlin.de.attr.major:computer\_sience". Each attribute-value pair gets a secret key assigned. The public component for this attribute will be published on the public bulliton board of the AA. 

In the next step, users are registered to the server. They receive the GID and a certificate for this GID. They can use the certificate later on to register to the AA or to authenticate themself to other users. The GID will have the form "<AID>.user.<UID>" where UID is a universal unique identifier (\ac{UUID}) in version 4.  
User receive their secret attribute keys by retrieving them from their AA. The AA verifies the certificate and issues the user his roles.

In the final setup, users can issue each other two factor keys so called \textit{authentication keys}. This authentication keys introduce a new layer of security where users can decide who exactly can decipher their plaintext. In some cases this may be needed since access policies in ABE describe alsways a group of user. 
If Bob wants to get an authentication key from Alice he sends a authentication request to Alice containing this certificate. Alice checks the validity of the certificate and returns the Bob-specific authentication key.

The previous steps are summarized in the figure \ref{fig:tfdacmacs-setup}.

\subsection{Encryption}
\begin{figure}[!t]
\centering
    \includegraphics[width=\linewidth]{img/TF-DAC-MACS-overview-encrypt.png}
    \caption{Encryption phase}
    \label{fig:tfdacmacs-encrypt}
\end{figure}

To upload a file encrypted under an access policy Alice first encrypts the file symetrically to create a file key (figure \ref{fig:tfdacmacs-encrypt}). This results in an file key and an encrypted file. The encrypted file is split up into different file chunks. Alice requests from the server a signed upload URL which she can use to upload the chunks to the CSPs. 

The file key, on the other hand, will be encrypted with an access policy defined by Alice. In addition she is also able to encrypt the ciphertext with the authentication key she issued to Bob. This encrypted file key will be uploaded to server where it is stored securly. 

\subsection{Decryption}
\begin{figure}[!t]
\centering
    \includegraphics[width=\linewidth]{img/TF-DAC-MACS-overview-decrypt.png}
    \caption{Setup phase}
    \label{fig:tfdacmacs-decryption}
\end{figure}

On decryption first the encrypted file key will be downloaded from the server. If Bob has a matching super set of attributes he can decipher it using his secret attribute keys. He might also apply the two factory key from Alice to endup with the plain file key (figure \ref{fig:tfdacmacs-decryption}. 

Next, he downloads the file chunks from the CSP, assembled them back together and decryphs them with the revocered file key.

\subsection{Revoke attribute}
\begin{figure}[!t]
\centering
    \includegraphics[width=\linewidth]{img/TF-DAC-MACS-overview-revoce-attr.png}
    \caption{Attribute revocation}
    \label{fig:tfdacmacs-attr-revocation}
\end{figure}

As displayed in figure \ref{fig:tfdacmacs-attr-revocation}, a revokation of an attribute key is always triggered by the AA that administers this attriubte. It frist creates a new secret key for the revoked attribute and calculates for each user owning the old attribute a delta. This delta is send to each non revoked user respectivly. In addition, the AA calucates a cipher text update key. This is send to the server which then in turn starts updating all the cipher text for the new attribute secret key. This operation will not affect the plaintext message in any kind. 

Alice and Bob, both receiving their update key, calculate their new attribute secret key. The AA can publish the new attribute public key on its public bulliton board.

\subsection{Revoke two-factor key}
\begin{figure}[!t]
\centering
    \includegraphics[width=0.6\linewidth]{img/TF-DAC-MACS-overview-revoce-user-key.png}
    \caption{Two-factor key revocation}
    \label{fig:tfdacmacs-user-auth-key-revocation}
\end{figure}

In a similar way as it was done for attribute revocation, Alice now starts to calcuate a new two-factor key. She computes the delta to all old two-factor keys she issed and distributes them to all non-revoced users. Finally, she calculates the ciphertext update key and sends it to the server. The server updates all relevant cipher texts.


\subsection{Adaptions and Improvements}
While TF-DAC-MACS satisfy all the requirementes it fits not perfect. To make the scheme more usable, the fixed two-factor authentication was removed and the key generation of the attribute which was defined in the setup phase as implied that the values need to be known at beginng, was moved to the point where the user requests a new attribute. Further, we propose a simple technique to break up TF-DAC-MACS m-of-n treshold policy with the traidoff for performance. And finally, we a technique proposed in \cite{bethencourt2007ciphertext} to implement numerical values and comparisons in boolean access formular. 

\subsubsection{Removing the fix two-factor contrain}
To make \ac{TF-DAC-MACS} more pratically applicable we removed the fixed two-factor contrain from the encyrption, decryption, and cipher update part. The two factor identifier $\alpha$ is used by the data owner to restirct the access to the content to certain users. 

This leads to the fact that the underlying \ac{ABE} schemes looses some of it expressivness. The zero knowledge of the data owner on which invidual is able to decrypter the cipher text is broken with the two factor part. Here each user that wants to decryper the encrypted text need to make an \textit{authentication request} to the data owner to receive the corresponing decryption key. To restore the possiblity to let an unkown user group decrypter the cipher text, we removed the two factor part. To do so we adapted encryption, decryption and cipher text update. The authentication key update will be ignored since it makes no sense to apply it on a non exisiting authentication key. 

\begin{itemize}
\item \textbf{Encryption:} 
We only need to update the $C_3$ part of the cipher text since it is the only one containing the two factor component $\alpha$.

The original $C_3$:
$$
C_3 = \Big( \prod_{v_{aid_{i}, j}\in W} g^{y_{aid_{i}, j}} \Big)^{s + \alpha} 
$$
is adapted to:
$$
\widehat{C}_3 = \Big( \prod_{v_{aid_{i}, j}\in W} g^{y_{aid_{i}, j}} \Big)^s
$$ 
In addition we will remove $oid$ from the ciphertext describtion since it refferese to the data owner ID. Which is only needed on authentication key update.

\item \textbf{Decryption:}
$SK_W = \prod_{v_{aid_i,j} \in W} SK_{v_{aid_i,j}}$ and $UPK_W = \prod_{v_{aid_i,j} \in W} UPK_{v_{aid_i,j}}$ remain defined in the same was as defined in the paper. 

On decryption the user does not need to generate $UPK_W$ and $SK_{uid, oid}$ anymore. The decryption equation is updated to:

$$
m = \frac{C_1 \cdotp e(H(uid), \widehat{C}_3)}{e(C_2, SK_W)}
$$

Note that the original decryption equation results in the above equation when the two factor part is deducted.

\begin{equation}
\begin{split}
m &= \frac{C_1 \cdotp e(H(uid), C_3)}{e(C_2, SK_W)e(SK_{uid, oid}, UPK_W)} \\
  &= \frac{C_1 \cdotp e\Big(H(uid), \Big( \prod_{v_{aid_{i}, j}\in W} g^{y_{aid_{i}, j}} \Big)^{s + \alpha} \Big)}{e(C_2, SK_W)e(H(uid)^\alpha, \prod_{v_{aid_i,j} \in W} UPK_{v_{aid_i,j}})} \\
  &= \frac{C_1 \cdotp e\Big(H(uid), \Big( \prod_{v_{aid_{i}, j}\in W} g^{y_{aid_{i}, j}} \Big)^{s + \alpha} \Big)}{e(C_2, SK_W)e(H(uid)^\alpha, \prod_{v_{aid_i,j} \in W} g^{y_{aid_i,j}})} \\
  &= \frac{C_1 \cdotp e\Big(H(uid), \Big( \prod_{v_{aid_{i}, j}\in W} g^{y_{aid_{i}, j}} \Big) \Big)^{s + \alpha}}{e(C_2, SK_W)e(H(uid), \prod_{v_{aid_i,j} \in W} g^{y_{aid_i,j}})^\alpha} \\
  &= \frac{C_1 \cdotp e\Big(H(uid), \Big( \prod_{v_{aid_{i}, j}\in W} g^{y_{aid_{i}, j}} \Big) \Big)^{s}}{e(C_2, SK_W)} \\
  &= \frac{C_1 \cdotp e\Big(H(uid), \Big( \prod_{v_{aid_{i}, j}\in W} g^{y_{aid_{i}, j}} \Big)^{s} \Big)}{e(C_2, SK_W)} \\
  &= \frac{C_1 \cdotp e(H(uid), \widehat{C}_3)}{e(C_2, SK_W)}
\end{split}
\label{eq:2faRemoval}
\end{equation}

As shwon, no security is threadned since we end up at the same equation as we would if we had the two factor part included. 

\item \textbf{Attribute revocation:}
The cipher text update key is adapted from

$$
CUK^{ID_W}_{v_{aid_i,j}} = (g^s \cdotp g^\alpha)^{y'_{aid_i,j} - y_{aid_i,j}}
$$

to 

$$
\widehat{CUK}^{ID_W}_{v_{aid_i,j}} = (g^s)^{y'_{aid_i,j} - y_{aid_i,j}}
$$

$\widehat{C}'_3$ now computes as 

\begin{equation}
\begin{split}
\widehat{C}'_3 &= \widehat{C}_3 \cdotp \widehat{CUK}^{ID_W}_{v_{aid_i,j}} \\
&\cdotp \Big( \prod_{v_{aid_{t}, j}\in W, v_{aid_t, j} \neq v_{aid_i,j}} g^{y_{aid_{i}, j}} \Big)^{r} \cdotp (g^{y'_{aid_i,j}})^{r} \\
&= \Big( \prod_{v_{aid_{t}, j}\in W, v_{aid_t, j} \neq v_{aid_i,j}} g^{y_{aid_{i}, j}} \Big)^{s + r} \cdotp (g^{y'_{aid_i,j}})^{s + r}
\end{split}
\end{equation}

It can be shown that $C'_3$ computes to the message $m$ in the same way as shown in eqation \ref{eq:2faRemoval}.

\item \textbf{Authentication update:}
Nothing need to change since cipher text do not contain authentication components. 
\end{itemize}

\subsubsection{Dynamic secret key generation}
Another small tweak in the \ac{TF-DAC-MACS} scheme was that the attributes for each \ac{AA} do not have to be known on AA inizialisation. They can be creates as well on each users key gen. This reduces the universe of possible attribute values to thouse who are actually needed.

\subsection{Extension to m-of-n treshold policy}
Extending to m-of-n treshold policy is also quite straight forward if we are willing to make traide offs in performance. So the client could upload different versions of the plain text encrypted under different policies. A client only need to decipher one of thouse cipher texts to revocer the message.

\subsection{Numerical boolean comparisons}
As described in \cite{bethencourt2007ciphertext} we could display numeric values in binary. Each number $x$ is composed of $\lceil log_2(x) \rceil$ attributes. Each of this attributes relate to either a $1$ or $0$ in one position in the binary number respresentation of $x$. So for example the number $5$ in binary would be displayed as $0101$ and its attribute would be: $x:0***$, $x:*1**$, $x:**0*$ and $x:***1$. 

If a user now wants to create a policy where he challanges a number $x$ to be greater or equal to $3$ he would create a policy: "$(x:1*** or (x:*1** or (x:**1* and x:***1))$". Analogous the policy for $x$ smaller than $4$: "$(x:0*** and (x:*0** or (x:*1** and x:**0* and x:***0))$".

Disadventage of this representation is of cause that it is limited in space. To display a 32-bit number we must issue 64 attriubte values and maintain 64 attriubte value keys. 


\subsection{Technologies}
To develop the first prototype of the system defined previously we will use the following technology stack:

\begin{itemize}
	\item \textbf{Spring boot}
	\item \textbf{Docker}
	\item \textbf{jPBC} \cite{ISCC:DecIov11}
	\item \textbf{...}
\end{itemize}
\todo{write me}

\subsection{Problems}

\subsubsection{En- and decrypting arbitrary data}
TF-DAC-MACS takes as an input for encryption a message $M \in GT$. Since there is not easy way to reconstruct a message from an element in $GT$, we have to combine some encryption teachniques to encrypt arbitrary data. 

The alogirhtm first chooses a random $M \in GT$ and outputs $M$ together with the constructed cipher text. $M$ is then hashed into a byte buffer using \ac{SHA}-256. In the next step we will encrypt our arbitrary data using \ac{AES} and as a key the hash previous computed. 

On decryption we frist reconstruct $M$ using the ABE decryption technique and then hashing $M$ again to reconstruct our AES secret key. This will help us to decipher the data that was encrypted. 



% Evaluation
\chapter{Evaluation}
\label{sec:evaluation}

The main issue of the current implementation in Bdrive is the poor scalability of the number of file keys. For each new device, a new file key needs to be created and maintained. The proposed prototype serves as the proof-of-concept that such a distributed, secure system can exist and which in addition fits all the requirements of section \req{sec:requirements} and scales better than the current system implemented in Bdrive. 

To stress this thesis, different benchmarks will be conducted to compare the proposed prototype to a similar environment such as Bdrive. The goal of this section will be to show that following assumption holds true:

\begin{center}
\textit{The ABE-based solution of the in section \ref{sec:implementation} proposed system to solve the scalability problem of a secure cloud storage system, scales at least as good as the RSA-based approach, described in section \ref{sec:background}, regarding the number of file keys that need to be maintained.}
\end{center}

While this assumption should hold true for all group operations the performance of thous operations need to be taken into account as well. The goal is to find the number of attributes per user, at which \name performs at least as good as the RSA-based sibling. It is expected that such a number can be found for encryption and member join operations. But when it comes to member leave operations, it is expected that there is no configuration in which the proposed prototype performances better then the RSA solution, due to additional shared-key managed and complex revocation mechanism. 

\section{Upper-Bound and Worst-Case Scenario}
\label{sec:upper-bound-and-worst-case-scenario}
In this section it will theoretically argued why \name scales at least as good as the current solution for secure cloud storage systems. The number of file-keys scale in a ratio of one-to-one with the number of "OR"-gates $+ 1$\footnote{A file key need to be created for the left and right sub-tree respectively. Having an access policy with one "OR"-gate will case the creation of two file keys.} in the access policy as described in section  \ref{sec:extension-to-dnf-policy}. The maximal number of "OR"-gates in an access policy would be to mention each user in the group explicitly. Such a policy looks like the following example:

\begin{center}
\begin{lstlisting}[caption={Worst case access policy. The used email address is a unique attribute per user.},captionpos=b]
(alice@example.com OR bob@example.com OR ... OR zara@example.com)
\end{lstlisting}
\end{center}

It does not make sense to include more "OR"-gates into this policy since it would represent redundant information. The number of files keys per $n$ user scale with $O(n)$ (same way as Bdrive) in \name if and only if, no suitable subset of district attributes of any two users can be found so that the $n$ users can be described with $o < n-1$ "OR"-gates $o$. 
That means if at least $k$-users, with $k > 1$  can be completely described by an AND-policy, the number of file keys needed will shrink to $n – k +1$ file keys. Each "OR"-gate that can be substituted by "AND"-gates reduces the number of file keys. 
If that is not the case, say no user share the same attribute, for each user an unique attribute needs to be used (for example the email address of the user) which is embedded in an inclusive attribute policy. 

Analogous, the best access policy that can be constructed would be one using only "AND"-gates resulting in only one file-key regardless of the number of embedded attributes. This behavior is inherited from  TF-DAC-MACS where it is possible to embed any number of attributes in an disjunctive condition into one file-key. 

\section{Performance and Scalability Analysis}

To evaluate the performance and scalability of \name against the RSA-based implementation different benchmarks are conducted. While the main focus remains on reducing the number of file keys for group operations, it is also important to measure the performance of encryption, member join and member leave actions. In the following sections each of the three different actions will be evaluated and an expectation about the experiment outcome will be given.

The benchmark evaluation is executed on a Server machine utilizing 6 vCPU cores (Intel(R) Xeon(R) CPU E5-2680 v3 @ 2.50GHz, 30MB cache size), 12 GB RAM and running on Ubuntu 16.04 LTS 64bit. For the evaluation OpenJDK 1.8.0\_191 is used. Each of the benchmark does not spawn any threads or uses parallelism. \footnote{Single thread assumption is based on the implementation of \name. This assumption must not hold true for libraries that are used under the hood for RSA and AES cryptography (bouncycastle 1.60) or pairing-based cryptography (jPBC 2.0.0).}  For pairing operations the native PBC extension is used. \footnote{\url{https://crypto.stanford.edu/pbc/}} 

Each benchmark averaged over 25 executions. Still artifacts and noise may occur, casing spikes or sudden outliers. Before each benchmark a warm-up phase is executed to fill caches and optimize branch predictions. However, it is still possible graphs become linear decreasing over time. Such behavior can only be traced back to optimized caches and less left-over artifacts from the initialization process.

\subsection{Encryption}
Encryption is the most interesting topic to evaluate since it scales using RSA with the number of users and using ABE with the number of attributes. With having the assumption of section \ref{sec:comparing-secure-group-communication-to-attribute-based-encryption} in mind, that states that the number of attributes is smaller or equal to the number of users in the shared group, ABE should be able to archive a better performance. 

\subsubsection{Expectation}
Classical encryption that uses RSA to encrypt the file key asymmetrically scales with the number of key-identities, in the following entitled users. For each new file in the group a suitable file key for each user must be created. The resulting assumption is that the RSA based approach scales linearly with increasing number of users. Same holds true for the cipher text size, which describes the aggregated sum of all file keys for one file. The number of file keys should scale on the order of 1-to-1 with the number of users. 

In contrast to that stands \name. It scales with the number of attributes per cipher text rather then the number of users. Given this assumption an intersecting point of "attributes per user" can be calculated where \name scales better then the RSA approach. Where this intersecting point is located depends on the number of attributes, number of users, the chosen policy and the computing power of the executing machine. At some point the access policy can get so large that the prototype solution will have to advantage regarding the computation time. 

\subsubsection{Benchmark: "AND"-Policy}
There where two different scenarios evaluated. The and-policies (best-case) and then the or-policies (worst-case) are analyzed independently from each other for performance and scalability. The benchmarks are performed over an increasing number of users. For each $n$ users a new attribute is introduced. The notation of $a$-for-$n$ defining $a$ as the number of attributes \textit{for} $n$ users. The benchmark is evaluated for the configurations $1$-for-$\infty$, $1$-for-$1$, $1$-for-$2$, $1$-for-$4$, $1$-for-$5$. Given the number of users the attributes can be calculated in the following way: $\lfloor \frac{n}{a} \rfloor$. 

% picture
\begin{figure}[!t]
\centering
    \includegraphics[width=\linewidth]{img/eval-and-policy/encrypt_incrementing_10.png}
    \caption{$1$-for-$\infty$ configuration using "and"-policy}
    \label{fig:1-for-infty-and}
\end{figure}
\begin{figure}[!t]
\centering
    \includegraphics[width=\linewidth]{img/eval-and-policy/encrypt_incrementing_10_attribute_increment_1per5User.png}
    \caption{$1$-for-$5$ configuration using "and"-policy}
    \label{fig:1-for-5-and}
\end{figure}
\begin{figure}[!t]
\centering
    \includegraphics[width=\linewidth]{img/eval-and-policy/encrypt_incrementing_10_attribute_increment_1per1User.png}
    \caption{$1$-for-$1$ configuration using "and"-policy}
    \label{fig:1-for-1-and}
\end{figure}

The configuration of $1$-for-$\infty$ as shown in figure \ref{fig:1-for-infty-and} describes the best-case scenario. Here a group can be completely described by only one attribute regardless of the number of users. \name has an constant overhead since the number of attributes remain constant. The intersecting point can be approximated at $145$ users. From that point on wards \name scales better than the RSA-based approach. With increasing the number of attributes per user, the overhead of \name becomes greater. In the $1$-for-$5$ configuration in Figure \ref{fig:1-for-5-and} the overhead of the prototype has become linear. This is due to additional computations for the increasing number of attributes in the cipher-text. In the configuration of $1$-for-$1$  (Figure \ref{fig:1-for-1-and}), so that each new user introduces a new attribute, the intersecting point gets pushed back to roughly $200$ users. 

Comparing the number of file keys over the increasing number of users, it is observable that the number of file keys indeed scale truly linearly to the number of users in the RSA-based solution. On the other hand, number of file keys remain constant at $1$ for \name when using only "AND"-gates. The cipher text size raises linearly with increasing number of attributes that need to be embedded into this policy. But due to less file keys it still scales better then the RSA based approach. 

\subsubsection{Benchmark: "OR"-Policy}
\begin{figure}[!t]
\centering
    \includegraphics[width=\linewidth]{img/eval-or-policy/encrypt_incrementing_10_attribute_increment_1per1User.png}
    \caption{$1$-for-$1$ configuration using "OR"-policy}
    \label{fig:1-for-1-or}
\end{figure}
\begin{figure}[!t]
\centering
    \includegraphics[width=\linewidth]{img/eval-or-policy/encrypt_incrementing_10_attribute_increment_1per2User.png}
    \caption{$1$-for-$2$ configuration using "OR"-policy}
    \label{fig:1-for-2-or}
\end{figure}
\begin{figure}[!t]
\centering
    \includegraphics[width=\linewidth]{img/eval-or-policy/encrypt_incrementing_10_attribute_increment_1per16User.png}
    \caption{$1$-for-$16$ configuration using "OR"-policy}
    \label{fig:1-for-16-or}
\end{figure}
\begin{figure}[!t]
\centering
    \includegraphics[width=\linewidth]{img/eval-or-policy/encrypt_incrementing_10_attribute_increment_1per140User.png}
    \caption{$1$-for-$140$ configuration using "OR"-policy}
    \label{fig:1-for-140-or}
\end{figure}

Completely different picture is drawn for the benchmark of "OR"-policies. Since for each "OR" a new cipher text needs to be created \name scales with the number of "OR"-gates present in the access policy. This affects execution time, number of file-keys and cipher-text size.


%\begin{figure}[!t]
%\centering
%    \includegraphics[width=\linewidth]{img/eval-or-policy/encrypt_incrementing_10_attribute_increment_1per200User.png}
%    \caption{$1$-for-$200$ configuration using "OR"-policy}
%    \label{fig:1-for-200-or}
%\end{figure}


Please note that the RSA-based function is the same plot in every figure. The scale and the ABE-plot differs for each figure.  

The worst-case scenario, as introduced in section \ref{sec:upper-bound-and-worst-case-scenario}, is defined as the scenario were the number of users is equal the number of unique attributes combined in an "OR"-policy. This case is shown in figure \ref{fig:1-for-1-or}. In that case is the encryption time of \name two magnitudes higher ($\approx 0ms-550$ms) than the computation time of the RSA-approach ($\approx 0ms-10$ms). Even the cipher text size is substantial greater than the cipher text size per file key. This results from the fact that now, \name scales with the same number of file keys as the RSA approach. No advantage can be gained by the prototype in the worst-case scenario. In general it holds true that the regarding the cipher text size and the number of file keys \name scales better if a group of user can be described by $o < n -1$ "OR" conditions $o$ then there are $n$ users.

As expected, if a configuration of one attribute per two users is used (Figure \ref{fig:1-for-2-or}) the file keys scale twice as good and the cipher text size scale rather similar to the size of the RSA based file keys. 

Reducing the overhead to a 1-for-16 configuration using "OR"-policies, as shown in Figure \ref{fig:1-for-16-or}, a step wise, increasing function is visible in each of the plotted graphics. Each step is directly correlated to another attribute in the access policy.

Finally, in Figure \ref{fig:1-for-140-or}, a configuration was chosen where \name is expected to scale better then the RSA-based approach. Based on that benchmark, the assumption can be extracted that \name scales better regarding computation time if only each 140 users a new OR-gate is introduced in the group access policy.  

\subsection{Member Join}
On member join, in RSA each file key needs to be reencrypted for the new member. That means that each file-key needs to be decrypted using the private key of an existing member and encrypted again with the new members public key. \name has an advantage since it just need to issue the new member the according secret user attribute keys. 

\subsubsection{Expectation}
Based on those facts, it can be assumed that \name scales an constant overhead while the RSA approach scales linearly with the increasing number of cipher texts. While the number of file keys will stay the same for \name, for the reference system they will scale depending on the number of cipher texts linearly. 

\subsubsection{Benchmark: Member Join}
\begin{figure}[!ht]
\centering
    \includegraphics[width=\linewidth]{img/eval-join/join_attr_1.png}
    \caption{Rekeying for one new member. Scaled over the number of attributes}
    \label{fig:member-join}
\end{figure}

Three group configurations were evaluated: On having 1, 2 or 4 attributes respectively. In the benchmark it is assumed that a user need to be issued all of the 1, 2 or 4 attribute secret keys that describe the group to become part of it. As shown in figure \ref{fig:member-join} the previously stated assumptions can be backed. Each of the different \name configurations show a constant overhead. Depending on the number of attributes the intersecting points of 20, 40 and 90 cipher texts can be extracted. After this amount of cipher texts in the group the prototype scales better.

\subsection{Member Leave}
Member-Leave operations comes with a greater overhead for \name. In that scenario, to revoke a member from a group, all of the group attributes need to be revoked from him. That means that for each group attribute, a new attribute private, public and for each non-revoked user an secret update key need to be created by the AA. The secret update keys need to be applied to the secret attribute keys of the non-revoked users. And finally, a cipher text update key needs to be calculated and applied to all cipher texts that contain the revoked attribute to make them accessible to the new attribute key.

The RSA-based approach has an big advantage here: Due to the fact that each file key serves as a unique entry point for a specific user to decrypt the encrypted file, to revoke the user form the group this file key just have to be deleted. 

\subsubsection{Expectation}
The assumption is simply stated that \name does not scale better then the RSA-based approach. Since it scales with the number of attributes that need to be revoked $a_{rev}$, the number of other non-revoked users $u-1$ and the number of cipher texts $c$ we end up with an overhead of $O(a_{rev}(u-1)c)$ which equals a cubic overhead. The reference system is expected to just scale linearly with the number of cipher texts. 

\subsubsection{Benchmark: Member Leave}
\begin{figure}[!ht]
\centering
    \includegraphics[width=\linewidth]{img/eval-leave/leave_attr_1_users_2.png}
    \caption{One member leaves the share. Scaled over the number of attributes}
    \label{fig:member-leave}
\end{figure}

Figure \ref{fig:member-leave} shows the exact expected behavior. For already one cipher text the gab between the different number of attributes and the RSA-based approach is clearly notable. \name is magnitudes slower than the RSA-based approach.

In a real world system the cipher text and the user secret key update would happen on the server and client device side respectively. Further, the application of the update key to the secret keys or cipher texts can happen in parallel to further speed-up this process. In the end \name scales much worse since it comes with unneglectable more overhead. 

\section{Summary}
The previous performance and scalability analysis strengthen the thesis that \name scales better than the reference implementation of the classical RSA-based sharing scheme. It was shown that in the worst-case scenario the same number of file-keys where produced and in the best-case scenario only one file-key need to be maintained. While the number of file-keys are certainly an issue, the size of the resulting file-keys need to be respected as well. The proposed ABE-based solution consumes, due to additional meta information, such as the access policy and data owner references, a much more storage per file-key. Calculating this into account \name archives its goal of consuming less file-key storage if a group can be described by $n/2$ conditions combined using "ORd-gates (see figure \ref{fig:1-for-2-or} for reference).

In the end it boils down to the access policy a user chooses. If this policy is constructed in an excluding fashion (using only or mostly "AND"-gates)  \name scales better in the long-term. But if to much attributes are used in the access policy many cipher-texts need to be updated each time a user or attribute gets revoked from the system. 



% Conclusion
\chapter{Conclusion}

\begin{figure}[!t]
\centering
    \includegraphics[width=1.0\linewidth]{img/share_distribution_bdirve.png}
    \caption{The distribution of shared folders per users and their aggregated devices.}
    \label{fig:evaluation-share-distribution}
\end{figure}

\section{Future adaptations}

\todo{
* Data owner Id can relate to the selected two factor key. 
* This would enable us to have multible keys 2FA keys per user. each securing an own CT. 
* make ownerId -> 2FA id 

* Handle attributes on device level and not on user level 
}

%----------------------------------------------------------------------------------------
%	THESIS CONTENT - APPENDICES
%----------------------------------------------------------------------------------------

%\appendices % Cue to tell LaTeX that the following "chapters" are Appendices

\chapter*{Acronyms}
\addcontentsline{toc}{chapter}{Acronyms}
\printacronyms[heading=none]


%----------------------------------------------------------------------------------------
%	BIBLIOGRAPHY
%----------------------------------------------------------------------------------------
\printbibliography[heading=bibintoc]


%----------------------------------------------------------------------------------------

\end{document}  
