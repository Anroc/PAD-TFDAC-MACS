%%%%%%%%%%%%%%%%%%%%%%%%%%%%%%%%%%%%%%%%%
% Journal Article
% LaTeX Template
% Version 1.4 (15/5/16)
%
% This template has been downloaded from:
% http://www.LaTeXTemplates.com
%
% Original author:
% Frits Wenneker (http://www.howtotex.com) with extensive modifications by
% Vel (vel@LaTeXTemplates.com)
%
% License:
% CC BY-NC-SA 3.0 (http://creativecommons.org/licenses/by-nc-sa/3.0/)
%
%%%%%%%%%%%%%%%%%%%%%%%%%%%%%%%%%%%%%%%%%

%----------------------------------------------------------------------------------------
%	PACKAGES AND OTHER DOCUMENT CONFIGURATIONS
%----------------------------------------------------------------------------------------

\documentclass[twocolumn]{article}

\usepackage{blindtext} % Package to generate dummy text throughout this template 

\usepackage[sc]{mathpazo} % Use the Palatino font
\usepackage[T1]{fontenc} % Use 8-bit encoding that has 256 glyphs
\linespread{1.0} % Line spacing - Palatino needs more space between lines
\usepackage{microtype} % Slightly tweak font spacing for aesthetics


\usepackage[english]{babel} % Language hyphenation and typographical rules

\usepackage[hmarginratio=1:1,top=32mm,columnsep=20pt]{geometry} % Document margins
\usepackage[hang, small,labelfont=bf,up,textfont=it,up]{caption} % Custom captions under/above floats in tables or figures
\usepackage{booktabs} % Horizontal rules in tables

\usepackage{lettrine} % The lettrine is the first enlarged letter at the beginning of the text
\usepackage[titletoc,toc,title]{appendix}

\usepackage{enumitem} % Customized lists
\setlist[itemize]{noitemsep} % Make itemize lists more compact

\usepackage{abstract} % Allows abstract customization
\renewcommand{\abstractnamefont}{\normalfont\bfseries} % Set the "Abstract" text to bold
\renewcommand{\abstracttextfont}{\normalfont\small\itshape} % Set the abstract itself to small italic text

\usepackage{titlesec} % Allows customization of titles
\renewcommand\thesection{\Roman{section}} % Roman numerals for the sections
\renewcommand\thesubsection{\roman{subsection}} % roman numerals for subsections
\titleformat{\section}[block]{\large\scshape\centering}{\thesection.}{1em}{} % Change the look of the section titles
\titleformat{\subsection}[block]{\large}{\thesubsection.}{1em}{} % Change the look of the section titles

\usepackage{fancyhdr} % Headers and footers
\pagestyle{fancy} % All pages have headers and footers
\fancyhead{} % Blank out the default header
\fancyfoot{} % Blank out the default footer
\fancyhead[C]{Practical MA-ABE for Secure Cloud Storage Systems $\bullet$ August 2018 $\bullet$ Vol. I, No. 2} % Custom header text
\fancyfoot[EL]{\thepage} % Custom footer text

\usepackage{titling} % Customizing the title section

\usepackage{hyperref} % For hyperlinks in the PDF
\usepackage{multicol}
\usepackage{graphicx}
\usepackage{caption}
\usepackage{amsmath}
\usepackage[textsize=footnotesize]{todonotes} % for todo notes

% define new command for \req (bold and italic)
\newcommand{\req}[1]{\textbf{\textit{#1}}}

% use \todo[inline] per default
\let\originaltodo\todo
\renewcommand{\todo}{\originaltodo[inline]}

\newcommand{\question}[1]{\originaltodo[color=green, inline]{#1}}

%----------------------------------------------------------------------------------------
%	TITLE SECTION
%----------------------------------------------------------------------------------------

\setlength{\droptitle}{-5\baselineskip} % Move the title up

\pretitle{\begin{center}\Huge\bfseries} % Article title formatting
\posttitle{\end{center}} % Article title closing formatting
\title{Practical Multi-Authority Attribute-Based Encryption Scheme for Secure Cloud Storage Systems } % Article title
\author{%
\textsc{Marvin Petzolt}\\[1ex] % Your name
\normalsize TU Berlin \\ % Your institution
\normalsize \href{mailto:marvin.petzolt@protonmail.com}{marvin.petzolt@protonmail.com} % Your email address
%\and % Uncomment if 2 authors are required, duplicate these 4 lines if more
%\textsc{Jane Smith}\thanks{Corresponding author} \\[1ex] % Second author's name
%\normalsize University of Utah \\ % Second author's institution
%\normalsize \href{mailto:jane@smith.com}{jane@smith.com} % Second author's email address
}
\date{\today} % Leave empty to omit a date
\renewcommand{\maketitlehookd}{%
\begin{abstract}
If a group of users want to share a ciphertext, they need to agree on a shared key. In order to ensure forward and backward secrecy, this shared key needs to be updated and distributed to every member everytime a user leaves or joins the group. Classical encryption methods encounter their limits, in which this approach is no longer feasible for a large number of users. In this work the field of attribute based encryption (ABE) will be analyzed to solve this issue for a secure cloud storage environment (Bdrive). Different ABE schemes will be compared for their applicability in Bdrive. Finally, the implementation of the resulting scheme will be evaluated for performance, scalability and security.
\end{abstract}
}

%----------------------------------------------------------------------------------------

\begin{document}

% Print the title
\twocolumn[
    \maketitle
]


%----------------------------------------------------------------------------------------
%	ARTICLE CONTENTS
%----------------------------------------------------------------------------------------

\section{Introduction}
\label{sec:introduction}
\todo{
	* **Introduction**\\
	* Ruff problem describtion\\
	* How ABE can solve this\\
	* How ABE works\\
	* Short history summary of ABE\\
	* What is expected of this work\\
	* What are the targets
}
\lettrine[nindent=0em,lines=3]{I}n public-key cryptography each user is identified by his unique public and private key pair. Peer-to-peer communication works well with this scheme, but as soon as an encrypted content needs to be accessed by multiple participants, the data owner has to encrypt the content for each user explicitly. This results in many encrypted versions of the same  file, each secured by a different public key. Such scheme does not scale in the secure cloud storage domain. Here often data holders want to share the same content with many other users at the same time.

We reach the point where the classical public-key end-to-end encryption scheme does not scale anymore. We would like to employ an encryption scheme which has a constant number of access keys regardless of the number of participants.

\section{Related Work}
\todo{
	* **Related work**\\
	* Of paper did something similar to this\\
	* Comparison paper\\
	* Not ABE paper
}

\todo{
	Cite \cite{lee2013survey} and similar overview paper
}

\subsection{Attribute-Based Encryption}
Attribute Based Encryption (ABE), first introduced by Sahai and Waters \cite{sahai2005fuzzy}, is a cryptographic encryption scheme which uses attributes of a user as keys for encryption. This enables the data owner to craft a ciphertext over chosen attributes that can only be decrypted by any entity that holds a super set of matching attributes. Further, it is possible to embed a complex access structure (e.g. access trees) inside the cipher text, where each node contains \textit{AND} or \textit{OR} threshold gates. This approach is known as Ciphertext-Policy Attribute-Based Encryption (CP-ABE) \cite{bethencourt2007ciphertext}. 
It is also possible to do it the other way around: Associate the user's key with an access policy, formally known as Key-Policy Attribute Based Encryption (KP-ABE).\cite{goyal2006attribute}. 
%Now the encryptor needs only to encrypt a given plain text with the public key of specific attributes so that only users who hold the right keys are able to decrypt the cipher text.

Both approaches are limited to one attribute authority (AA). So only one trusted entity is in charge of issuing attributes and their matching key pairs. However, in the real world we would like to distribute the issuing of attributes over different authorities. A new scheme is needed where different attribute authorities cooperate and communicate across different domains.  

% The basic idea is, as proposed in \cite{chase2007multi}, to construct for each users a polynomial of degree $d-1$, where $d$ donates the number of attributes in our system. 

Multi-Authority Attribute Based Encryption (MA-ABE), first proposed by Chase 2007 \cite{chase2007multi}, allows multiple attribute authorities to maintain different attribute domains. To ensure collusion resistance between users, a trusted  Central Authority (CA) was introduced to assign each user an unique identifier (UID) and making the decryption process depending on this UID. Its disadvantage is the CA's global decryption power and therefore the necessity to remain trusted. 
%To prevent collusion between user in a multi authority setting, the challenge for each user needs to be individual. But it still needs to be ensured that the encryption of a message is independent of any user specific identifier, since the encryption progress should sourly depend on the attribute set known to the system.
%To mitigate this problem a global identifier (GID or UID) per users was introduced that is shown to each attribute authority (AA) to receive the corresponding private key for the users attributes. 
%The central authority (CA) now has to make sure that the user dependent challenge results in a global decryption key to decrypt the message.
%In fact the CA has to be trusted since it computes the users private keys in such a way that on decryption it reveals the global decryption key. 
Chase \textit{et al.} addressed this issue in \cite{chase2009improving} by distributing the global secret  master key generation over the AAs. However, since the global master secret is computed on system initialization, no further AAs can be added afterwards without rebooting the system. Also the scheme did not support attribute revocation which rendered it practical not applicable.
% Each authority uses this seed in combination with a pseudo random function to deterministically create the users private key. Since the CA possesses the same seed, same pseudo random function and issues the users GID, it can also compute the same private key as issued by the corresponding AA. So it can ensure that on decryption the keys add up to reveal the global decryption key. This scheme has a major issue: The CA has global decryption power. 

% Chase notices this problem as well and tried to mitigate it by decentrializing the seed generation \cite{chase2009improving}.  First user attributes could not easly be revoked\footnote{Chase proposed to assign each user a range of GIDs, so that a user can migrate to the next GID on revocation. Since the GID range is finite, this solution does not scale.}. Second AA's cant be added on runtime without reissuing each user his keys. And last, the attribute universe is finite (no large-universe). Further we noticed that if a failure of one AA renders the system unresponsible, since a cyphertext includes attributes from each authority. 

Data Access Control for Multi-Authority Cloud Storage Systems (DAC-MACS) \cite{yang2013dac} is a scheme that tackles both of this issues. Each authority receives its own ciphertext that depends solely on the attributes issued in this domain. While Chase managed to maintain the "one-plaintext-one-ciphertext" ratio, DAC-MACS needs to create $k$ ciphertexts: One per AA.\footnote{If the ciphertext does not require any attributes of an specific authority it does not have to create a ciphertext for this domain.} It does not require any coordination between authorities which enables to add new AAs at runtime without recreating the user keys. This scheme also includes features for efficient revocation while it claims to maintain forward and backward secrecy. 

DAC-MACS is not collusion resistance on attribute revocation under the active attack model. The scheme NEDAC-MACS (New Extended DAC-MACS) shows and solves this vulnerability \cite{wu2017security}. Recent studies introduce a more efficient, scalable and secure approaches such as MAACS \cite{li2016secure} and TFDAC-MACS (Two-Factor DAC-MACS)\cite{li2017two}. The DAC-MACS family is further known for the proxy decryption technique, first introduced by \cite{li2013matrix}, where a honest-but-curious server helps the user on decryption.

Priscilla and Nagarajan 2018 \cite{nagarajan2018overview} conducted an analysis of the DAC-MACS family. In addition to that work, this thesis will focus on alternative approaches to improve scalability of secure cloud storage systems and analyze the field of ABE and MA-ABE more deeply.

%------------------------------------------------

\section{Contribution}
\todo{
	* What are the targets\\
	* What makes this work stand out\\
	* What we can gain from this work
}
The target of this work is to find a more scalable solution to the currently enrolled re-keying scheme in Bdrive. The new scheme needs to satisfy the requirements of section \ref{sec:requirements} with respect to multi-company management (each company administrates only its own domain) and inter-company file sharing (it is possible to share files across companies), as well as it should be practically applicable in the real world.

%------------------------------------------------

% Background  / Problem description
\input{./content/05-background-bdrive}

\input{./content/06-requirements}

% Secure group communication
\input{./content/07-secure-group-communication}

% solution: ABE
\input{./content/08-introducation-to-the-field-of-attribute-based-encryption}

% ABE
\section{Basic Attribute-Based Encryption}
In the following sections we will habe a look at the different attribute based encryption schems. We will extract characterisitics of thouse schemes to cluster them, select a repectable candidate from each of those schemes and to finally, make a practical performance anaylsis of thouse schemes. 

A suitable candidate in the field of attribute based encryption scheme is a scheme that satisfies the requirements stated in section \ref{sec:requirements}. The requirements are a super set of thouse defined in \cite{lee2013survey}. For the basic ABE schemes we will focus on the requirements \req{C1}, \req{C3}, \req{C5} - \req{C8} and optional requirements \req{O1} and \req{O2}.
The general requirements \req{B1} and \req{B2} will be evaluated by pratical a performance and scalability anaylsis.

\subsection{Collusion resistance (C2 requirement)}
Lets construct a very basic attribute based encryption scheme to clarify the importance of collusion resistance. Assume baisc RSA cryptography. We setup an AA which combines attributes to RSA public-private key pairs. Attribute "student" gets bound to $K_{PR(s)}$ for the private key and $K_{PU(s)}$ for the public key. Attribute "works at TU Berlin" gets bound to the Key $K_{PK_PR(tu)}$ and $K_{PU(tu)}$ respectivly. Now, we setup our very own first ABE scheme. The AA can pin the public keys of each attributes to its public billboard so that every entity in the system can encrypt for thouse attributes. Each user who is currently a student receives a copy of the private key $K_{PR(s)}$ and each user who is currently working at TU Berlin receives a copy of $K_{PR(tu)}$. 

A user, lets call her for historical reason Alice, wants to share content with all students that are also working at TU berlin. She encrypts the plaintext $p$ with both public attribute keys $c = K_{PU(s)}(K_{PU(TU)}(P))$ and publishes the ciphertext $c$ to a public CSP so that everyone can download it. Students that are working at TU Berlin owning both private keys can decipher the ciphertext by applying both private keys in reverse order $K_{PR(TU)(K_{PR{s}}(c)}$.

The attentive reader will have notice a cural security leak in this scheme. This ABE scheme is not cullusion restiant. In fact, collusion restistance is a core requirement of any attribute-based encryption schemem. On paper it is defined as the impossibility of any two attribute holder to combine their attributes to archive a higher level of encryption. Lets assume that Bob is a student and Eve is working at TU Berlin. Both users received their private key. Now they can simply exchange their attribute keys so that they are both able to decipher the ciphertext even if they severativly don't own both attributes.  

Usally collusion restance is ensured by issuing each user a private key that is blinded by a random value. This random value will vanish on decryption. However, if two users collude they will mix their blinded values resulting a plaintext that is still by some unknown value. 

\subsection{Bilinear Mappings}
To ensure collusion resistance bilinnear maps are comonly used. They are a tool for \textit{cytrographic pairings} and define the relationship between cyclic groups of the same (prime) order. A cyclic group $G$ is defined by an generator $g$ of that form that each $g^n \in G$ with $n \in \mathbb{Z}$ complelty desribes each element in the group $G$.

A bilinear map function $e$ is now defined as the mapping between two groups of the same order $G_1$ and $G_2$ into $G_t$:

$$ e : G_1 \times G_2 \rightarrow G_t $$
$$\text{ such that for all } g_1 \in G_1, g_2 \in G_2, a, b \in \mathbb{Z}$$
$$e(g_1^a, g_2^b) = e(g_1, g_2)^{ab}$$

%In some schemes $G_1$ and $G_2$ relate to the same cyclic group. Another nice fact is that the \textit{Decisional Bilinear Diffie-Hellman Problem} becomes easy computable. 

This contruct of pairings is used by many schmes to either distribute a secret or computing the secret without ever revealing it. To now ensure collusion resistance with this technique normally a user specific id (UID) is generated and bound to the exponent of the user's attribute private key. The decryption method will be designed in that way that the blinding of the attribute private key will vanaish.   

\subsection{KP-ABE}
In Key-Policy Attribute-Based Encryption (KP-ABE) \cite{goyal2006attribute}, is a ABE technique that assoizates chiphertexts with attributes that fullfill the policy embedded in the key addressed to a user. 

To evaluate if a policy matches given attributes an \textit{Access Tree} or \textit{Linear Secrect Sharing Matrix} (LSSS) is used. While both representations represent the same boolean formular a LSSS is often more efficient. For a better explanation the model of an access tree will be used. For each node of this tree it is evaluated whether the children satisfy a certain condition. This might be implemented as \textit{OR} or \textit{AND} treshhold gates. While \textit{OR} can be expressed as $1$-out-of-$n$ children need to satisfy the condition, \textit{AND} conditions are $n$-out-of-$n$ threshhold gates meaning all children need to satisfy the condition. This approach is a monotonic access strcture and often refered as \textit{Threshhold Security} or \textit{Treshhold Access Strcutre}. 

The initial paper did leck some basic requirements that was improved by other work. Such as a fix-length cipher text, attribute revocation. Further, the authors stated that users, onces issued a policy, are able to delegate a subset of their access tree to other users. Users themself could act as a new attriubte authority to issue other uses a subset of thier access policy. While this is a nice feature for some use cases, in a cloud system the company would like to restrict this delegation or would like to be able to track traidors selling their private keys as a blackbox decryption.

This encryption scheme was ment as a broadcasting encryption scheme. Used by radio providers so that they can address any user that bought a paid membership to distribute premium content. However, since the policy was bound to the users key it is not possbile to address a ciphertext to a specific user group without knowing who exaclty persesses the right policy. Only the central authority, which issued the users privayt keys in the first place, would know who could decipher the encrypted text. This fact renders KP-ABE impractical for tha application in a cloud sharing scheme. User would simply don't know who are able to decrypter the cipher text. 

The certainly very specific use case of KP-ABE probably led to the circumstances that not much paper using this technique exists. Even so that we could not find any paper satisfying all requirements previously specified. We found only paper that implemented direct revocation using negation of attributes \cite{lewko2010revocation}. This would in turn lead to the fact that users private key sizes would grow with each revoced attribute. 

\subsection{CP-ABE}
On the other side there is Cipher-text policy Attribute-Based Encryption (CP-ABE) \cite{bethencourt2007ciphertext}. CP-ABE assigns ciphertexts with a policy and user keys with attributes. This enables CP-ABE to be much more flexible and verbose in comparison to KP-ABE. The big adventages was (similar to KP-ABE) that the need for a central authentication server was obsulte. Each ciphertext could simple accessed by each user and only thouse who have the right attribute set present are able to decipher the ciphertext. 

The lack of revocation \req{C8} and the need for a large attribute and user universe was huge \req{C5}. Revocation means that a system manager could revoke attributes from users or even the user as such. In modern company settings an required feature to ensure forward secrecy. Large attribute universe on the other hand describes that the attributes that can be distributed by the AA are somewhat so large that it is unlikly that a running system will run out of attributes to issue. 

The first proposal of a revocation scheme in CP-ABE was in 2010 \cite{liang2010ciphertext}. This scheme used a similar approach like LKH to make the revocation process effient. In 2016 Lui \textit{et. al} \cite{liu2016practical} proposed a recovable ABE scheme that supported traidor tracing and large attribute universe. However, both scheme were restricted to a fix number of users or a limited attribute universe. Which rendered them pratical not applicable.  

\subsection{Non-monotonic access strcutre of ABE}
While monotonic access structures can describe \textit{AND} or \textit{OR} gates, non-monotonic access structure can also refere to \textit{NOT} gates. This adds another layer of find-ganylarity into the system. While in monotonic access strctures user get an attribute assigned they have access to all topics addressed to this attribute. In non-monotonic strctures, first intoduced by Ostrovsky \textit{et. al.} \cite{Ostrovsky:2007:AEN:1315245.1315270}, certain topics may be excluded. 

Take for example Alice who wants is an intern in the Top Secret company. In monotinic access strcutre she would receive both attribute private keys from "Intern" and "working in Top Secret Company". Since than she would be able to decipher all content addressed to all employees at the Super Secret company. However, some content might be so confidential that interns shell not be able to access them. This exclusion would not be possible. With non-monotonic access structure an administrator may want to encrypt certain information with the policy "working in Top secret Company AND NOT intern".  

While being an imported field in the boolean formular and fine-grand access control domain, non-monotonic access structures did not gain the same attention as CP or KP-ABE did. As an candidate which shell represent this ABE topic \cite{10.1007/978-3-642-54631-0_16} CP-ABE implementation will be used. With negation of attributes revocation becomes more or less trivial. Each attribute can be versioned and simply exlcuded on encryption. While this approach is not scalable over time it is simple enough to be implemented in some schemes.   

\subsection{Multi-Authority Attribute Based Encryption}
Ontop of the previos mentioned schemes emerged a new sub topic of ABE: \textit{Multi-Authority Attribute Based Encryption} (MA-ABE). The main motivation was that a single AA is not practically applicable. In the real world we face different domain each maintaining it own attributes. 

MA-ABE while beeing a roughly young field of ABE encryption technique enjoyed a lot of attention in the research area. In addition to the normal requirements like collusion resistance and revocation mechanism, MA-ABE also deals with the question on how to deescalate the global decryption power of the central authority (CA). 

A setup of a MA-ABE system looks quite similar over the field of related work. On system inizialisation we setup the CA. The purpose of the CA is to bootrap new Attribute Authoritieis (AA) which then will administer their domain. In this domain users and attributes of the AA are located. After the CA bootstraped all AAs and all users are registed it could in theority go offline. 

To ensure system wide collusion resistance each user usally gets a unique user identifiery (UID) assigned. This UID is issued by the CA which is the only entity having an overview about the whole system. 

A suitable condidate to compare MA-ABE to the previous introduced schemes is \textit{hirachical attribute based encryption} (HABE, section \ref{sec:HABE}). In short HABE is structured around the idea of attribute administration delegation. The strcture can be imaged like the domain name system. On top level there is the root master administating the whole domain. He can delegate power in form of attribute issuing und user setup to sub entities. This sub entities can again forward a subset of their power to their childen. For a complete explenation see \ref{sec:HABE}. We will use \cite{Wang:2010:HAE:1866307.1866414} as representiv candidate for MA-ABE. 

\subsection{Comparison}
To compare the selected representived of the previous section we choose the charm framework\footnote{\url{http://charm-crypto.io/}}. Here we can already find implementations for the sutable KP-ABE candidate \cite{lewko2010revocation} and for non-monotonic CP-ABE \cite{10.1007/978-3-642-54631-0_16}. The other two schemes such as CP-ABE \cite{liu2016practical} and \cite{wang2011hierarchical} where implemented by us in a github frok of the charm repo\footnote{\url{https://github.com/Anroc/charm}}. 

\begin{table*}[!ht]
\centering
\begin{tabular}{l 					| l 				| l 				| l 				| l}
									& \textbf{LSW 08}	& \textbf{YAHK 14}	& \textbf{LW 14}	& \textbf{WLWG 11} 	\\
\req{Scheme}						& KP				& Non-Monotone CP 	& CP 				& Hirachical CP		\\ 
\req{Security scheme}				& Biliniear maps 	& Binilnear maps 	& Biliniear maps 	& Biliniear maps 	\\
\req{Expression of access policy}	& LSSS				& LSSS matrix 		& LSSS matrix 		& DNF 				\\ 
\end{tabular}
\caption{Scheme description. }
\label{tab:comparison_baic_abe_overview}
\end{table*}
\begin{table*}[!ht]
\centering
\begin{tabular}{l 	| l					| l 				| l 				| l}
					& \textbf{LSW 08}	& \textbf{YAHK 14}	& \textbf{LW 14}	& \textbf{WLWG 11} 	\\
\req{C1}			& Yes				& Yes 				& Yes 				& Yes 				\\
\req{C2}			& No				& No 				& No 				& Yes 				\\ 
\req{C3}			& No				& No 				& No 				& No 				\\ 
\req{C4}			& No				& No 				& No 				& No 				\\ 
\req{C5}			& Yes				& Yes 				& Yes 				& Yes 				\\ 
\req{C6}			& - 				& - 				& -					& Yes				\\
\req{C7}			& -					& - 				& - 				& No 				\\
\req{C8}			& Yes				& Yes				& Yes				& Yes				\\
\req{O1}			& No 				& No 				& Yes 				& No 				\\
\req{O2}			& Yes+ 				& Yes+				& Yes				& Yes-				\\
\end{tabular}
\caption{}
\label{tab:basic_abe_comparisons}
\end{table*}

In table \ref{tab:comparison_baic_abe_overview} the different scheme techniques are listed.  And in Table \ref{tab:basic_abe_comparisons} this schemes are compared to the requirements of section \ref{sec:requirements}. 

From table \ref{tab:basic_abe_comparisons} it is clear that no scheme fully fulfill all requirements. However, they reveal a good indicator where we should look next. WLWG 11 clearly makes the race here satisfying 6 of 10 of the requirements. 

\req{C6} and \req{C7} could not be sutisfied by LSW08, YAHK14 or LW14 since they are not designed to support multible authorities. 
\question{Should I respect MA-ABE also here?}
Lets have a look at \req{O2}. LSW08 and YAHK14 both supports fine grant access controll by using a LSSS access structure. Also both implement non-monotonic access structes which give them the extra plus. LW14, on the other hand, does not support negation of attributes, but support an arbirary access strcture like LSW08 and YAHK14. WLWG11, however, does only support access strcutures in DNF form which makes this scheme somehow restriced in expressivness. 

\begin{figure*}[!ht]
\centering
    \includegraphics[width=1\linewidth]{img/basic_abe_comparisons.png}
    \caption{Performance and scalability comparison between the chosen schemes with increasing number of attributes. The policy for each new attribute $a_k$ with $a_k \in A$ was defined as $\bigwedge\limits_{a \in A}^a a$. Key generation relates to the creation of users key pairs given an accesspolicy or attribute set.}
    \label{fig:basic_abe_comparison}
\end{figure*}

Notable about the comparison in figure \ref{fig:basic_abe_comparison} is that while KP-ABE (LSW08) was expected to have the biggest overhead in Keygeneration compared to the CP-ABE schemes and the lowest in encyrption, since no policy has to be parsed and computed here. However, none of the both assumtions was true. It seams more like the time complexlity and runtime of the different algorithms suerly depends on the design of the scheme. 

Also remarkable is that WLWG11 has the best overall performance. But it must also be noticed that in the we did not include the attribute authoirity key generation. While the paper claimed to have an constant overhead on decryption it is impressive to see that it holds true. 

While the plaintext message remained contant it is clearly visible that the ciphertext length groth linearly with the number of attributes. 

Due the greate performance and the coverage of the most of the requirements of the hirachical attribute based encryption technique, the field of multi-authority attribute based encyrption will be evaluated in more depth. Regardless of the great scalability of the HABE it comes with a non neglegtable disadventage. The root master and the top level domain master have both global decryption power. For each user regardless of the company affiliation the administator of the system will always be able to dechipher ciphertexts of thouse users. This would break end-to-end encryption complelty. So clearly we would favor solutions that support different attribute authorities so that each company can administer their own domain seperatly from each other, but we want also that root, while beeing able to boostrap new attribute authorities, is not able to decihper any ciphertext. 




% MA-ABE
\section{Multi-authority attribute based encryption}
In this section the field of Multi-authority attribute-based encryption (MA-ABE) will be evaluated more deeply with respect to all requirements of section \ref{sec:requierements}. MA-ABE while beeing a roughly young field of ABE encryption technique enjoyed a lot of attention in the research field. 
In addition to the normal requirements like collusion resistance and revocation mechanism, MA-ABE also deals with the question on how to deescalate the global decryption power of the central authority (CA). 

A setup of a MA-ABE system looks quite similar over the field of related work. On system inizialisation we setup the CA. The purpose of the CA is to bootrap new Attribute Authoritieis (AA) which then will administer their domain. In that domain are users and attributes of this AA located. After the CA bootstraped all AAs and all users are registed it could in theority go offline. 

To ensure system wide collusion resistance each user usally gets a unique user identifiery (UID) assigned. This UID is issued by the CA which is the only entity having an overview about the whole system. 

Another notable technique recently used by \cite{yang2013dac}, \cite{wu2017security}, \cite{li2017two} and \cite{wang2011hierarchical} is \textit{proxy de-/reencryption}. It is motivated by the fact that mobile devices often don't have much computing resources so that the server may help the clients for decryption. The main idea is that the server does the preprocessing of the encrypted text given paramteres by the user. In the end the preprocessed encrypted content will be easily computable by the edge device. The cural fact is that the server will have no knowledge about the plaintext but rather helps the user on decryption or reencryption. 

In the following the different sub topics of MA-ABE will be described, analyzed given the requirement set and finally evaluated based on their perforamnce and scalability. 

\subsection{Introduction into Multi-Authority Attribute-based Encryption}
Chase 2007 \cite{chase2007multi} was the frist known to introduce the first MA-ABE schemes. In her paper she describes the process on how to derrive a multi-authority attribute-based encryption scheme vom the single authrotity schemes. 

The collussion resistance criteria in ABE implies that any data owner needs to encrypt under attributes in such kind that the ciphertext is independent of any users specific identifiers. Any user could in theory decrypter the ciphertext if he holds the fullfilling attribute set. While chase's scheme also used bilinear maps it main security was based on interpolation and the fact that no underdefined linaer equation system could definitly be solved\footnote{For LSSS matrix are based on the same assumption.}.  

Colluding users can not decrypter any cipher text and the plain text is encrypted using a blinded master secret. The blinding factor is then implemented in the policy so that any user satisfying this policy can restore the blinding factor and so revocer the plaintext. 

In the next step each user is given blinded attribute private keys so that when used on decryption the plaintext is still blinded with the user specific identifier. Using pairings this identifier can be substituted and the plaintext gets revealed. 

Two cural disadventages where not respected by chase in her initial scheme. First the CA had global decryption power. It needed to issue each AA a specific seed so that on using this seed in a pheudo random generator the CA knew what random value would be calculated. This was importent since the CA need to isse the user his private key which combined with the private attribute keys from the AA would reveal on decryption the unblinded plaintext. 

Chase improved her scheme 2009 in \cite{chase2009improving}. Now all AAs will do a n-party key exchange to distribute the secret seeds with each other. Agreeing on a well known value the CA was no longer needed and as long $N-2$ AAs are not colluding with each other the secret remained secure. 

The other disadventage, that was will present in the updated scheme, was that no new AA could be added after system inizialisation since it would trigger a new system inizialisation and distribution of the master secret. 

The lack of addition new attribute authorities and the missing revocation scheme made chases scheme impractical for further evaluations but give a good introduction the fitfalls of MA-ABE schemes.

\subsection{Decentralized attribute-based encryption}

\subsection{Efficient Data Access Control for Multi-Authority Cloud Storage}
The most explored field in MA-ABE is the Efficient Data Access Control for Multi-Authrotity Cloud Stroage (DAC-MACS) family \cite{yang2013dac}. First instroducted by Yang \textit{et. al.} 2013 it desribes an efficient, revocable MA-ABE scheme using proxy encryption on decryption and reencryption. Furhter this scheme features a large attribute universe, adding AAs on the running system and serves with a CA that has no global decryption power. In sort DAC-MACS satisfy all the non-optional requirements.

DAC-MACS elliminates the need for the global decryption power of the CA by issuing $k$ ciphertexts: One per AA. \footnote{If the ciphertext does not require any attributes of an specific authority it does not have to create a ciphertext for this domain.} It does not require any coordination between authorities which enables to add new AAs at runtime without recreating the user keys. This scheme also includes features for efficient revocation while it claims to maintain forward and backward secrecy.

DAC-MACS is not collusion resistance on attribute revocation under the active attack model. The scheme NEDAC-MACS (New Extended DAC-MACS) shows and solves this vulnerability \cite{wu2017security}. Recent studies introduce a more efficient, scalable and secure approaches such as MAACS \cite{li2016secure} and TFDAC-MACS (Two-Factor DAC-MACS)\cite{li2017two}.


% Implementation
\section{TF-DAC-MACS}
\todo{write section}

\subsection{Adaptions and Improvements}
To make TF-DAC-MACS more pratically applicable we removed the fixed two-factor contrain from the encyrption, decryption, and cipher update part. The two factor identifier $\alpha$ is used by the data owner to restirct the access to the content to certain users. 

This leads to the fact that the underlying ABE schemes looses some of it expressivness. The zero knowledge of the data owner on which invidual is able to decrypter the cipher text is broken with the two factor part. Here each user that wants to decryper the encrypted text need to make an \textit{authentication request} to the data owner to receive the corresponing decryption key. To restore the possiblity to let an unkown user group decrypter the cipher text, we removed the two factor part. To do so we adapted encryption, decryption and cipher text update. The authentication key update will be ignored since it makes no sense to apply it on a non exisiting authentication key. 

\begin{itemize}
\item \textbf{Encryption:} 
We only need to update the $C_3$ part of the cipher text since it is the only one containing the two factor component $\alpha$.

The original $C_3$:
$$
C_3 = \Big( \prod_{v_{aid_{i}, j}\in W} g^{y_{aid_{i}, j}} \Big)^{s + \alpha} 
$$
is adapted to:
$$
\widehat{C}_3 = \Big( \prod_{v_{aid_{i}, j}\in W} g^{y_{aid_{i}, j}} \Big)^s
$$ 
In addition we will remove $oid$ from the ciphertext describtion since it refferese to the data owner id. Which is only needed on authentication key update.

\item \textbf{Decryption:}
$SK_W = \prod_{v_{aid_i,j} \in W} SK_{v_{aid_i,j}}$ and $UPK_W = \prod_{v_{aid_i,j} \in W} UPK_{v_{aid_i,j}}$ remain defined in the same was as defined in the paper. 

On decryption the user does not need to generate $UPK_W$ and $SK_{uid, oid}$ anymore. The decryption equation is updated to:

$$
m = \frac{C_1 \cdotp e(H(uid), \widehat{C}_3)}{e(C_2, SK_W)}
$$

Note that the original decryption equation results in the above equation when the two factor part is deducted.

\begin{equation}
\begin{split}
m &= \frac{C_1 \cdotp e(H(uid), C_3)}{e(C_2, SK_W)e(SK_{uid, oid}, UPK_W)} \\
  &= \frac{C_1 \cdotp e\Big(H(uid), \Big( \prod_{v_{aid_{i}, j}\in W} g^{y_{aid_{i}, j}} \Big)^{s + \alpha} \Big)}{e(C_2, SK_W)e(H(uid)^\alpha, \prod_{v_{aid_i,j} \in W} UPK_{v_{aid_i,j}})} \\
  &= \frac{C_1 \cdotp e\Big(H(uid), \Big( \prod_{v_{aid_{i}, j}\in W} g^{y_{aid_{i}, j}} \Big)^{s + \alpha} \Big)}{e(C_2, SK_W)e(H(uid)^\alpha, \prod_{v_{aid_i,j} \in W} g^{y_{aid_i,j}})} \\
  &= \frac{C_1 \cdotp e\Big(H(uid), \Big( \prod_{v_{aid_{i}, j}\in W} g^{y_{aid_{i}, j}} \Big) \Big)^{s + \alpha}}{e(C_2, SK_W)e(H(uid), \prod_{v_{aid_i,j} \in W} g^{y_{aid_i,j}})^\alpha} \\
  &= \frac{C_1 \cdotp e\Big(H(uid), \Big( \prod_{v_{aid_{i}, j}\in W} g^{y_{aid_{i}, j}} \Big) \Big)^{s}}{e(C_2, SK_W)} \\
  &= \frac{C_1 \cdotp e(H(uid), \widehat{C}_3)}{e(C_2, SK_W)}
\end{split}
\label{eq:2faRemoval}
\end{equation}

As shwon, no security is threadned since we end up at the same equation as we would if we had the two factor part included. 

\item \textbf{Attribute revocation:}
The cipher text update key is adapted from

$$
CUK^{ID_W}_{v_{aid_i,j}} = (g^s \cdotp g^\alpha)^{y'_{aid_i,j} - y_{aid_i,j}}
$$

to 

$$
\widehat{CUK}^{ID_W}_{v_{aid_i,j}} = (g^s)^{y'_{aid_i,j} - y_{aid_i,j}}
$$

$\widehat{C}'_3$ now computes as 

\begin{equation}
\begin{split}
\widehat{C}'_3 &= \widehat{C}_3 \cdotp \widehat{CUK}^{ID_W}_{v_{aid_i,j}} \\
&\cdotp \Big( \prod_{v_{aid_{t}, j}\in W, v_{aid_t, j} \neq v_{aid_i,j}} g^{y_{aid_{i}, j}} \Big)^{r} \cdotp (g^{y'_{aid_i,j}})^{r} \\
&= \Big( \prod_{v_{aid_{t}, j}\in W, v_{aid_t, j} \neq v_{aid_i,j}} g^{y_{aid_{i}, j}} \Big)^{s + r} \cdotp (g^{y'_{aid_i,j}})^{s + r}
\end{split}
\end{equation}

It can be shown that $C'_3$ computes to the message $m$ in the same way as shown in eqation \ref{eq:2faRemoval}.

\item \textbf{Authentication update:}
Nothing need to change since cipher text do not contain authentication components. 


\end{itemize}


%----------------------------------------------------------------------------------------
%	REFERENCE LIST
%----------------------------------------------------------------------------------------
\clearpage
\bibliography{multi-authority-abe} 
\bibliographystyle{ieeetr}

\clearpage
\onecolumn
\appendices

%----------------------------------------------------------------------------------------

\end{document}
