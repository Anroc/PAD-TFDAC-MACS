\section{Goining Beyond the limitations: An introduction to Attribute-Based Encryption}

To go beyond the limitations of secure group communications a different approach to this problem is needed. Lets take a step back and look at groups communications from another perspective then from the cryptographic side. How are groups composed? How are they formed and how is chosen who participates? 

\subsection{Relationship between groups and attributes}
Group in general summarize a specific group of individuals. Specific in that kind, that key share a common feature. In a company, for example, groups are based on shared attributes such as "working in the same team", "working in the same company" or "dinkes coffee". The more common use case, at least in the businesss area, is that users want to distribute content to groups rather then people. This means that if Alice wants to send an encrypted file to the management office she does not care which manager decrypts her file. Everyone in the deparment should be able to do so. 

\subsection{Comparing SGC to ABE}

Comparing SGC to ABE is non tivial. We can not simply compare the numbers of keys, or messages that are exchanged between client and server. This is due to a different encryption technique used by ABE: paring. Pairing scales more or less with the overhead of RSA rather then ECC or block ciphers as stated by \textit{Galbraith et. al.} \cite{galbraith2008pairings} resulting in bad\todo[inline]{bader?} performance compared to SGC.  But in specific scenarios ABE use less keys to setup the group communication. In the following thouse secnarios will be described and analysed where this schemes differe and what use-cases both schemes address.

We can state that ABE is adventagious on scenarios where users address an unknown group of individuals. This can be some departments (e.g. police department of New York), colloquiums (e.g. faculity for Secure network communications), or general user groups (all employees working in the security research team). On the other hand, SGC is beneficous if an user wants to address specific individuals (share a file with Alice and Bob but no other members).\footnote{While ABE is still capable of addressing this use case it sufferes from the less efficient encryption technique.}

Attribute based encryption targets the rekeying problem focusing on attribute and groups rather then individuals. ABE reduces the number of keys needed by combining and resuing keys, while secure group communciation schemes need to create a new key per each group. The clue is that groups can be described by an attribute set and if another group shares the same set of attributes the same keys are reused. In secure group communication a new group is bounded to a new group key. ABE relocates this key creation to the setup phase. Depending on the target of application secure group communication or ABE will be more adventagious.

Lets define an scenario adventagous to ABE. Alice wants to share a file with all management members of the coffee company. Since she does not know the members in person, nor their email addresses, she simply creates a share with the group "management of coffee company". Alice only needs to retrieve the key of the management from the central server of the coffee company. This proceedure took Alice, one encryption and two tranmissions: one to retrieve the key and one to upload the encrypted text. 

If we apply the same scenario to SGC we face a problem. How to know which public keys belong to the executive officers? Alice need to check on the webside which people are in charge of the coffee company, to download thier public keys, encrypt the group key with their public keys, and upload the file and the GK for each manager. This took alice one lookup of the role to key mappings, $n$ downloads of public keys, $n$ encryptions of the GK for each member and one encryption of plain text, and two uploads of the cipher text and the group key.         

In conclusion is ABE more sutable for the use case that this work tries to implement. ABE does not only handle the encryption, but also provides an authentication service. Users are bound to roles and attributes which are tidly interleafed with the encryption scheme. Bdrive target audience are business which by nature have some kind of role managment and authorization mechanism embedded. Here ABE can enfold its true potential and outperform SGC not only in efficiency but also in additional security features. 


